\documentclass[12pt,a4paper]{article}

\usepackage[utf8]{inputenc}
\usepackage[T1]{fontenc}
\usepackage[german]{babel}
\usepackage{minted}
\usepackage{csquotes}
\usepackage{graphicx}
\usepackage[colorlinks=true, linkcolor=blue, urlcolor=cyan]{hyperref}
\usepackage[backend=biber,style=authoryear]{biblatex}

\addbibresource{bibliography.bib}

\begin{document}

% ------ Titelblatt ------
\begin{titlepage}
    \centering
    {\Large Hochschule für Musik Karlsruhe \\ [1em]}
    {\large Institut für Musikinformatik und Musikwissenschaft \\[6em]}

    {\Large \textbf{Bachelorarbeit} \\[2em]}

    {\LARGE \textbf{Neuer Medienservers für die Bibliothek der Hochschule für Musik Karlsruhe} \\[6em]}

    \begin{minipage}{0.9\textwidth}
        \raggedright
        \textbf{Autor:} Lennart Rathgeb \\
        \textbf{Matrikelnummer:} 13883 \\
        \textbf{Adresse:} Insterburger Straße 2, 76139 Karlsruhe \\
        \textbf{Studiengang:} Musikinformatik (HF), Musikwissenschaft (EF) \\
        \textbf{Erstleser:} Prof. Dr. Christoph Seibert \\
        \textbf{Zweitleser:} Daniel Höpfner \\
    \end{minipage}

    \vfill
    Karlsruhe, den \today
\end{titlepage}

% ------ Inhaltsverzeichnis ------
\newpage
\tableofcontents
\newpage

% ------ Abschnitte ------
\section{Notizen}

\begin{enumerate}
  \item Server-Setup
    \begin{itemize}
      \item Rechner mit Ubuntu (24.04.3 LTS, da längerer Support) als Betriebssystem aufsetzen.
        \begin{itemize}
          \item Download von \url{https://ubuntu.com/download/server} auf USB-Stick
          \item Stick einstecken, Rechner booten und Installation starten
            \begin{itemize}
              \item dafür beim Anschalten des Rechners \texttt{Shift} drücken
              \item darauf achten, dass der Rechner in Zukunft nicht über USB bootet
              \item Rechner per LAN-Kabel mit dem Netzwerk verbinden
            \end{itemize}
        \end{itemize}

      \item Docker installieren
        \begin{itemize}
          \item Quickinstall:
          \begin{minted}[linenos]{bash}
sudo apt update
sudo apt install docker.io
sudo systemctl start docker
sudo systemctl enable docker
          \end{minted}

          \item Von Docker-Webseite (\url{https://docs.docker.com/engine/install/ubuntu/}):
          \begin{minted}[breaklines, linenos]{bash}
# Add Docker's official GPG key:
sudo apt-get update
sudo apt-get install ca-certificates curl
sudo install -m 0755 -d /etc/apt/keyrings
sudo curl -fsSL https://download.docker.com/linux/ubuntu/gpg -o /etc/apt/keyrings/docker.asc
sudo chmod a+r /etc/apt/keyrings/docker.asc

# Add the repository to Apt sources:
echo \
"deb [arch=$(dpkg --print-architecture) signed-by=/etc/apt/keyrings/docker.asc] https://download.docker.com/linux/ubuntu \
$(. /etc/os-release && echo "${UBUNTU_CODENAME:-$VERSION_CODENAME}") stable" | \
sudo tee /etc/apt/sources.list.d/docker.list > /dev/null
sudo apt-get update
          \end{minted}

          \item Test mit:
          \begin{minted}[linenos]{bash}
sudo docker run hello-world
          \end{minted}
        \end{itemize}
    \end{itemize}

  \item Auswahl und Installation der Medienserver-Software
    \begin{itemize}
      \item Hier kommen die Softwareoptionen rein
      \item Installationsschritte …
    \end{itemize}

  \item Integration mit der bestehenden IT-Infrastruktur
    \begin{itemize}
      \item Netzwerk-Anbindung
      \item Datenbanken anbinden …
    \end{itemize}

  \item Benutzerverwaltung und Zugriffsrechte
    \begin{itemize}
      \item Nutzergruppen definieren
      \item Rechteverwaltung testen …
    \end{itemize}

  \item Testen und Fehlerbehebung
    \begin{itemize}
      \item Funktionstests
      \item Lasttests
      \item Bugfixing dokumentieren …
    \end{itemize}
\end{enumerate}
\newpage

\section{Einleitung}
In der Vergangenheit gab es bereits Projekte einen Medienserver für die Bibliothek der Hochschule für Musik Karlsruhe aufzusetzen. 
Diese Projekte wurden jedoch nie abgeschlossen oder die Systeme wurden nicht dauerhaft betrieben.
Es soll einen ersten Medienserver gegeben haben, jedoch nur mit limitiererter Funktionalität, 
sodass Frederik Schroff im Jahr 2015 als Masterarbeit einen verbesserten Medienserver inklusive Datenbank aufsetzte.
Dieser ist allerdings seit einigen Jahren nicht mehr in Betrieb.
Während meiner Arbeit als Tutor am Medienserver, war meine offizielle Aufgabe stehts nur CDs einzulesen, zu benennen und abzuspeichern.
Diese waren anschließend aber nicht abrufbar, sodass ich anfing die Arbeit zu hinterfragen.
Nachdem ich als Praxisarbeit bereits den Digitalisierungsprozess der CDs durch die Erneuerung des dafür verwendeten Skripts verbesserte, 
möchte ich nun den Medienserver neu aufsetzen, damit die angesammelten Daten auch genutzt werden können.

Ziel ist es den Server so zu gestalten, dass er einfach zu betreiben ist und über eine benutzerfreundliche Oberfläche verfügt.
Es soll ermöglicht werden, die digitalisierten CDs damit unter Beachtung rechtlicher Rahmenbedingungen zu streamen.

\subsection{Zur Dokumentation}
Die Dokumentation Gliedert sich in zwei Teile, wobei der erste Teil die praktische Umsetzung beschreibt
und der zweite Teil auf technische Hintergründe eingeht.
Sie dient dazu eine Anleitung zur Nutzung und Wartung des Medienservers zu geben 
und einige technische Hintergründe zu verschiedenen Themen aus dem Bereich Netzwerk 
und Serveradministration zu geben, 
um ein besseres Verständnis für die Funktionsweise eines Medienservers zu schaffen.
Zunächst wird im Kapitel zum Setup des Servers die einzelnen Schritte zur Installation und Konfiguration beschrieben.
Danach wird auf das Skript zur Datenkonvertierung und Metadatenvergabe eingegangen,
welches wichtiger Betanstandteil ist, um weitere Medien in den Server einzupflegen.


\section{Implementierung}
\subsection{Vorraussetzungen}
\subsection{Sicherheit}

\section{Technischer Hintergrund}
- Medienserver
- Docker
- Ubuntu Server
- Navidrome
- NAS

\section{Fazit}
% Fazit folgt ...


% Beispiel für Zitation
Wie in \textcite{kuster2005musikinformatik} beschrieben, ...
Weitere Informationen finden sich bei Navidrome \parencite{navidrome2024}.

\section{Literaturverzeichnis}
%Literaturverzeichnis folgt ...

\printbibliography

\end{document}
