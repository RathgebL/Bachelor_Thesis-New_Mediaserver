\documentclass[12pt,a4paper]{report}

\usepackage[utf8]{inputenc}
\usepackage[T1]{fontenc}
\usepackage[german]{babel}
\usepackage{minted}
\usepackage{csquotes}
\usepackage{graphicx}
\usepackage[colorlinks=true, linkcolor=blue, urlcolor=cyan]{hyperref}
\usepackage[backend=biber,style=authoryear]{biblatex}

\addbibresource{bibliography.bib}

\begin{document}

% ------ Titelblatt ------
\begin{titlepage}
    \centering
    {\Large Hochschule für Musik Karlsruhe \\[1em]}
    {\large Institut für Musikinformatik und Musikwissenschaft \\[6em]}

    {\Large \textbf{Bachelorarbeit} \\[2em]}

    {\LARGE \textbf{Neuer Medienservers für die Bibliothek der Hochschule für Musik Karlsruhe} \\[6em]}

    \begin{minipage}{0.9\textwidth}
        \raggedright
        \textbf{Autor:} Lennart Rathgeb \\
        \textbf{Matrikelnummer:} 13883 \\
        \textbf{Adresse:} Insterburger Straße 2, 76139 Karlsruhe \\
        \textbf{Studiengang:} Musikinformatik (HF), Musikwissenschaft (EF) \\
        \textbf{Erstleser:} Prof. Dr. Christoph Seibert \\
        \textbf{Zweitleser:} Daniel Höpfner \\
    \end{minipage}

    \vfill
    Karlsruhe, den \today
\end{titlepage}

% ------ Inhaltsverzeichnis ------
\cleardoublepage
\pagenumbering{roman}
\tableofcontents

% ------ Abbildungsverzeichnis ------
\cleardoublepage
\listoffigures
\addcontentsline{toc}{chapter}{Abbildungsverzeichnis}
\listoftables
\addcontentsline{toc}{chapter}{Tabellenverzeichnis}

% ------ Einleitung ------
\cleardoublepage
\pagenumbering{arabic}
\setcounter{page}{1}
\chapter*{Einleitung}
\addcontentsline{toc}{chapter}{Einleitung}
\setcounter{section}{0}
\renewcommand\thesection{\arabic{section}}

In der Vergangenheit gab es bereits Projekte einen Medienserver für die Bibliothek der Hochschule für Musik Karlsruhe aufzusetzen. 
Diese Projekte wurden jedoch nie abgeschlossen oder die Systeme wurden nicht dauerhaft betrieben.
Es soll einen ersten Medienserver gegeben haben, jedoch nur mit limitiererter Funktionalität, 
sodass Frederik Schroff im Jahr 2015 als Masterarbeit einen verbesserten Medienserver inklusive Datenbank aufsetzte.
Dieser ist allerdings seit einigen Jahren nicht mehr in Betrieb.
Während meiner Arbeit als Tutor in der Bibliothek, war meine offizielle Aufgabe stehts nur CDs einzulesen, zu benennen und abzuspeichern.
Diese waren anschließend aber nicht abrufbar, sodass ich anfing die Arbeit zu hinterfragen.
Nachdem ich als Praxisarbeit bereits den Digitalisierungsprozess der CDs durch die Erneuerung des dafür verwendeten Skripts verbesserte, 
möchte ich nun den Medienserver neu aufsetzen, damit die angesammelten Daten auch genutzt werden können.

Ziel ist es den Server so zu gestalten, dass er einfach zu betreiben ist und über eine benutzerfreundliche Oberfläche verfügt.
Es soll ermöglicht werden, die digitalisierten CDs damit unter Beachtung rechtlicher Rahmenbedingungen zu streamen.

\section{Zur Dokumentation}
Die Dokumentation Gliedert sich in zwei Teile, wobei der erste Teil die technischen Hintergründe beschreibt
und der zweite Teil auf die praktische Umsetzung in der Bibliothek eingeht.
Damit dient sie nicht nur dazu einen Einstieg in den Themenbereich Netzwerk und Serveradminastration zu geben, 
sondert bildet auch die Anleitung zur Nutzung und Wartung des Medienservers in der Bibliothek ab.

Im ersten Teil werden grundlegende Begriffe erläutert, 
die Entwicklung von Servern und Medienservern nachgezeichnet 
sowie deren Anwendungsbereiche dargestellt. 
Darauf aufbauend werden technische Aspekte der Serverarchitektur, 
der Administration und der Netzwerktechnik vorgestellt, 
bevor auf spezifische Eigenschaften von Medienservern eingegangen wird. 
Abschließend wird die besondere Rolle von Medienservern im 
Bibliothekskontext betrachtet. 

Der zweite Teil beschreibt die praktische Umsetzung des Projekts. 
Hier wird zunächst die Zielsetzung definiert und die notwendigen 
Voraussetzungen hinsichtlich Hard- und Software sowie des Netzwerks erläutert. 
Es folgt die detaillierte Installation und Konfiguration der 
benötigten Komponenten, die Integration in die bestehende IT-Infrastruktur 
sowie die Einrichtung der Benutzerverwaltung und Zugriffsrechte. 
Darüber hinaus werden Testverfahren und Strategien zur 
Fehlerbehebung dokumentiert. 

Den Abschluss der Arbeit bilden ein Fazit mit einer 
kritischen Reflexion des Projekts sowie ein Ausblick 
auf mögliche Weiterentwicklungen.

% ------ Theoretische Grundlagen ------
\chapter*{Theoretische Grundlagen}
\addcontentsline{toc}{chapter}{Theoretische Grundlagen}
\setcounter{section}{0}

\section{Einführung in Server und Medienserver}
Server bilden das Fundament moderner IT-Infrastrukturen.
Sie übernehmen zentrale Aufgaben bei der Bereitstellung von Diensten 
und der Verwaltung von Daten. 
Im Zusammenhang mit Medienservern steht vor allem die Speicherung, 
Organisation und Übertragung digitaler Inhalte wie Musik, 
Videos oder Bilder im Vordergrund. 

Medienserver ermöglichen es, solche Inhalte in Echtzeit zu streamen, 
ohne dass ein vollständiger Download notwendig ist. 
Ihre Entwicklung ist eng mit der allgemeinen Geschichte von Servertechnologien und Netzwerken verknüpft 
und reicht von klassischen Rechenzentren über den Aufstieg des Internets bis hin zu modernen Heim- und Cloudlösungen. 

Heute finden Medienserver Einsatz in unterschiedlichsten Bereichen - 
von der privaten Nutzung über Unternehmen bis hin zu Bibliotheken, 
die digitalisierte Bestände für ihre Nutzer zugänglich machen.

  \subsection{Begriffsdefinitionen}
  Als \emph{Server} bezeichnet man einen „Rechner, 
  der für andere in einem Netzwerk mit ihm verbundene Systeme bestimmte Aufgaben übernimmt 
  und von dem diese ganz oder teilweise abhängig sind“. \footcite{duden_server}

  In diesem Zusammenhang ist ein \emph{Netzwerk} ein Zusammenschluss von „zwei oder mehr Computern,
  die miteinander verbunden sind, um Daten elektronisch auszutauschen.“
  (Übersetzung nach \footcite{britannica_network})

  Ein \emph{Medienserver} ist eine spezialisierte Form eines Servers, 
  die darauf ausgelegt ist, digitale Medieninhalte wie Audio, Video oder Bilder zu speichern, 
  zu verwalten und über ein Netzwerk für Clients bereitzustellen.\footcite[Vgl.][S.~131 ff.]{steinmetz_multimedia}

  Unter \emph{Streaming} versteht man die Übertragung von Audio- bzw. Videodaten von einem Server auf einen Client, 
  wobei die Wiedergabe umgehend erfolgt.\footcite{gabler_streaming}
  Multimedia-Inhalten können so bereits während der Übertragung konsumiert werden, 
  ohne dass eine vollständige Speicherung der Daten auf dem Endgerät erforderlich ist.

  Server und Clients bilden damit die grundlegende Kommunikationsbasis in Netzwerken. 
  Medienserver stellen eine spezielle Ausprägung dar, 
  die insbesondere für die effiziente Bereitstellung von Streaming-Inhalten optimiert sind.

  \subsection{Historische Entwicklung}
  Die Entwicklung von Servern und Netzwerken ist eng mit der allgemeinen Geschichte der Computerkommunikation verknüpft. 
  Erste Formen von Rechnernetzen entstanden bereits Ende der 1950er Jahre mit dem ARPANET, 
  das als Vorläufer des heutigen Internets gilt 
  und den Grundstein für die paketorientierte Datenübertragung legte.\footcite[Vgl.][Kapitel~1.4.2]{tanenbaum_computernetworks}

  In den frühen 1970er Jahren zeigte sich, dass die bestehenden Protokolle des ARPANET nicht ausreichten, 
  um heterogene Netzwerke wie Funk- oder Satellitennetze zu integrieren. 
  Vor diesem Hintergrund entwickelten Vinton Cerf und Robert Kahn 1974 das Transmission Control Protocol (TCP), 
  das später in die heute gebräuchliche Protokollfamilie TCP/IP aufgeteilt wurde. 
  TCP/IP ermöglichte die zuverlässige Kommunikation über unterschiedliche Netzwerke hinweg 
  und bildete die Grundlage für die Interoperabilität moderner Netze.\footcite[Vgl.][Kapitel~1.5.1]{tanenbaum_computernetworks}
  Im Jahr 1983 wurde TCP/IP im ARPANET als verbindlicher Standard eingeführt,
  was gemeinhin als Geburtsstunde des modernen Internets gilt.\footnote{Vgl. \cite[S.~2]{postel_rfc801}; \cite{britannica_tcpip}.}

  In den 1980er und 1990er Jahren etablierte sich das Client-Server-Modell als dominierendes Paradigma für den Datenaustausch. 
  Hierbei übernehmen Server die Rolle zentraler Dienstleister, 
  während Clients auf deren Ressourcen zugreifen.\footcite[Vgl.][Kapitel~1.7.1 und 1.7.2]{kurose_networking} 
  Mit der zunehmenden Verbreitung des Internets wurden Server in Unternehmen, 
  Hochschulen und öffentlichen Institutionen zu zentralen Infrastrukturen.

  Parallel dazu entwickelte sich der Bedarf an multimedialer Datenübertragung. 
  Erste Streaming-Verfahren wie RealAudio (1995) ermöglichten die kontinuierliche Übertragung von Audioinhalten, 
  ohne dass die Daten vollständig gespeichert werden mussten. 
  Diese Technologie wurde später auf Video übertragen 
  und in standardisierte Protokolle wie RTP/RTSP überführt.\footcite[Vgl.][Kapitel~2.3]{steinmetz_multimedia}

  Seit den 2000er Jahren prägen Breitbandinternet und die Verbreitung mobiler Endgeräte die Nutzung multimedialer Inhalte. 
  Parallel dazu entwickelten sich adaptive Streaming-Technologien wie MPEG-DASH (Standardisierung 2012)
  und Apple HLS (2009), die eine effiziente und bandbreitenabhängige Übertragung von Audio- 
  und Videoinhalten ermöglichen.\footcite[Vgl.][]{iso_mpegdash, apple_hls}
  Medienserver sind in diesem Kontext zu zentralen Plattformen für die Speicherung und Bereitstellung digitaler Bestände geworden 
  und finden heute sowohl im privaten Umfeld als auch in Institutionen wie Hochschulen und Bibliotheken Anwendung.

  \subsection{Anwenwendungsbereiche}
  Server übernehmen zentrale Aufgaben in der IT, etwa als Datei-, Datenbank-, Web- oder E-Mail-Server. 
  Darüber hinaus kommen sie in Bereichen wie Virtualisierung, Cloud-Computing
  oder bei Sicherheitsdiensten (z.\,B. Authentifizierung, Firewalls) zum Einsatz.

  Medienserver bilden eine spezialisierte Kategorie: Sie speichern und verwalten
  digitale Audio-, Video- und Bildbestände und stellen diese über Netzwerke bereit.
  Einsatzfelder finden sich im privaten Umfeld (Streaming von Musik und Filmen),
  in Unternehmen (z.\,B. interne Schulungsvideos oder Marketingmaterial),
  im Bildungsbereich (Vorlesungsaufzeichnungen, E-Learning-Plattformen) sowie
  in Kulturinstitutionen wie Archiven und Bibliotheken, wo sie die Nutzung und
  Langzeitverfügbarkeit digitalisierter Sammlungen unterstützen.\footcite[Vgl.][Kapitel~2.3.3]{steinmetz_multimedia}

\section{Grundlagen der Serverarchitektur}
Serverarchitekturen umfassen sowohl Hardware- als auch Softwareaspekte. Hier werden die technischen Grundlagen erläutert, die für den Betrieb eines Medienservers erforderlich sind.

  \subsection{Hardware- und Softwareaspekte}
  Überblick über grundlegende Komponenten und Betriebssysteme.

  \subsection{Betriebssysteme für Server}
  Vergleich verschiedener Systeme wie Linux-Distributionen, Windows Server oder FreeBSD.

  \subsection{Virtualisierung und Containerisierung}
  Techniken wie Virtual Machines, Docker und Kubernetes zur flexiblen Bereitstellung von Diensten.

  \subsection{Dateisysteme und Speicherlösungen}
  NAS, SAN, RAID und deren Rolle in der Datenhaltung.

\section{Serveradministration}
Die Administration eines Servers umfasst grundlegende Prozesse der Benutzerverwaltung, Wartung und Absicherung.

  \subsection{Benutzer- und Rechteverwaltung}
  Konzepte für Rollen und Zugriffsrechte.

  \subsection{Dienste und Prozesse}
  Installation, Konfiguration und laufender Betrieb von Serverdiensten.

  \subsection{Monitoring und Logging}
  Überwachung von Serverzuständen und Protokollierung von Ereignissen.

  \subsection{Backup- und Recovery-Strategien}
  Methoden zur Datensicherung und Wiederherstellung.

  \subsection{Sicherheitsaspekte}
  Firewalls, Verschlüsselung, Authentifizierung und Zugriffskontrolle.

\section{Netzwerktechnik für Medienserver}
Netzwerke sind die Grundlage für den Zugriff auf Medienserver. Dieser Abschnitt behandelt die wichtigsten Konzepte und Protokolle.

  \subsection{Grundlagen der Netzwerktechnik}
  TCP/IP, Ports und DNS als Basis des Datenaustauschs.

  \subsection{Streaming-Protokolle}
  HTTP, RTSP, RTP, HLS und DASH für die Übertragung von Multimedia-Inhalten.

  \subsection{Leistungsanforderungen}
  Bandbreite, Latenz und Pufferung im Streaming-Betrieb.

  \subsection{Qualität und Sicherheit}
  Quality of Service (QoS) sowie Netzwerksicherheit mit TLS/SSL und VPN.

\section{Spezifisches zu Medienservern}
Medienserver unterscheiden sich durch Software, Formate und rechtliche Rahmenbedingungen. Dieser Abschnitt stellt relevante Systeme und Standards vor.

  \subsection{Typische Medienserver-Software}
  Plex, Jellyfin, Emby, Kodi, VLC-Streaming, Icecast und DLNA-basierte Systeme.

  \subsection{Datenformate und Standards}
\begin{itemize}
    \item Audioformate: MP3, FLAC, AAC, WAV
    \item Containerformate: MKV, MP4
    \item Metadatenstandards: ID3, Dublin Core (insbesondere im Bibliothekskontext)
\end{itemize}

  \subsection{Rechtefragen}
    \begin{itemize}
        \item Lizenzen und Lizenzmodelle
        \item Urheberrechtliche Rahmenbedingungen
        \item Bibliotheksrecht im Hochschulkontext
    \end{itemize}

\section{Medienserver im Bibliothekskontext}
Der Einsatz von Medienservern in Bibliotheken eröffnet neue Möglichkeiten für den Zugriff auf digitalisierte Bestände, stellt jedoch auch besondere Anforderungen an Recht, Technik und Nachhaltigkeit.

  \subsection{Digitalisierung von Beständen}
  Beispiel: Umwandlung und Speicherung von Audio-CDs in digitale Formate.

  \subsection{Zugriff und Rechteverwaltung}
  Einschränkungen des Zugriffs, etwa nur im Hochschulnetz oder über VPN.

  \subsection{Nachhaltigkeit und Langzeitarchivierung}
  Strategien zur dauerhaften Sicherung digitaler Medien.

  \subsection{Lokale vs.\ Cloud-basierte Lösungen}
  Vergleich der Vor- und Nachteile unterschiedlicher Implementierungsansätze.

% ------ Praktische Umsetzung ------
\chapter*{Praktische Umsetzung}
\addcontentsline{toc}{chapter}{Praktische Umsetzung}
\setcounter{section}{0}

\section{Zielsetzung}
  \subsection{Funktionalität}
  \subsection{Benutzerfreundlichkeit}
  \subsection{Wartbarkeit}
  \subsection{Sicherheit}
  \subsection{Rechtliche Rahmenbedingungen}
\section{Vorraussetzungen}
  \subsection{Hardware}
  \subsection{Software}
  \subsection{Netzwerk}
  \subsection{Rechtliche Rahmenbedingungen}
\section{Installation}
  \subsection{Ubuntu Server}
  \subsection{Docker}
  \subsection{Navidrome}
  \subsection{nginx}
  \subsection{SSH}
  \subsection{UFW}
\subsection{Konfiguration}
  \subsection{Ubuntu Server}
  \subsection{Docker}
  \subsection{Navidrome}
  \subsection{nginx}
  \subsection{SSH}
  \subsection{UFW}
\subsection{Datenbank}
  \subsubsection{Datenstruktur}
    Benennungskonvention
  \subsubsection{Dateiformate}
  \subsubsection{Metadaten}
    \begin{itemize}
      \item ID3-Tags
      \item Vorbis Comments
      \item FLAC-Metadaten
      \item MusicBrainz
    \end{itemize}
  \subsubsection{Datenverwaltung}
    Datenpflege
    Backup-Strategie
  \subsubsection{Datenquelle}
    Skript zur Digitalisierung
    Skript zur Metadatenanreicherung
    Externe Quellen
  \subsubsection{Datenimport}
\section{Integration in die bestehende IT-Infrastruktur}
  \subsection{Netzwerk-Anbindung}  
  \subsection{Datenbank anbinden}
\section{Benutzerverwaltung und Zugriffsrechte}
  \subsection{Nutzergruppen definieren}
\section{Testen und Fehlerbehebung}
  \subsection{Funktionstests}
  \subsection{Problemlösung}   


\chapter*{Fazit}
\addcontentsline{toc}{chapter}{Fazit}
\setcounter{section}{0}
% Fazit folgt ...

\section{Ausblick}
% Ausblick folgt ...}

% ------ Notizen (am Ende entfernen!) ------
\section{Notizen}

\begin{enumerate}
  \item Server-Setup
    \begin{itemize}
      \item Rechner mit Ubuntu (24.04.3 LTS, da längerer Support) als Betriebssystem aufsetzen.
        \begin{itemize}
          \item Download von \url{https://ubuntu.com/download/server} auf USB-Stick
          \item Stick einstecken, Rechner booten und Installation starten
            \begin{itemize}
              \item dafür beim Anschalten des Rechners \texttt{Shift} drücken
              \item darauf achten, dass der Rechner in Zukunft nicht über USB bootet
              \item Rechner per LAN-Kabel mit dem Netzwerk verbinden
            \end{itemize}
        \end{itemize}

      \item Docker installieren
        \begin{itemize}
          \item Quickinstall:
          \begin{minted}[linenos]{bash}
sudo apt update
sudo apt install docker.io
sudo systemctl start docker
sudo systemctl enable docker
          \end{minted}

          \item Von Docker-Webseite (\url{https://docs.docker.com/engine/install/ubuntu/}):
          \begin{minted}[breaklines, linenos]{bash}
# Add Docker's official GPG key:
sudo apt-get update
sudo apt-get install ca-certificates curl
sudo install -m 0755 -d /etc/apt/keyrings
sudo curl -fsSL https://download.docker.com/linux/ubuntu/gpg -o /etc/apt/keyrings/docker.asc
sudo chmod a+r /etc/apt/keyrings/docker.asc

# Add the repository to Apt sources:
echo \
"deb [arch=$(dpkg --print-architecture) signed-by=/etc/apt/keyrings/docker.asc] https://download.docker.com/linux/ubuntu \
$(. /etc/os-release && echo "${UBUNTU_CODENAME:-$VERSION_CODENAME}") stable" | \
sudo tee /etc/apt/sources.list.d/docker.list > /dev/null
sudo apt-get update
          \end{minted}

          \item Test mit:
          \begin{minted}[linenos]{bash}
sudo docker run hello-world
          \end{minted}
        \end{itemize}
    \end{itemize}

  \item Auswahl und Installation der Medienserver-Software
    \begin{itemize}
      \item Hier kommen die Softwareoptionen rein
      \item Installationsschritte …
    \end{itemize}

  \item Integration mit der bestehenden IT-Infrastruktur
    \begin{itemize}
      \item Netzwerk-Anbindung
      \item Datenbanken anbinden …
    \end{itemize}

  \item Benutzerverwaltung und Zugriffsrechte
    \begin{itemize}
      \item Nutzergruppen definieren
      \item Rechteverwaltung testen …
    \end{itemize}

  \item Testen und Fehlerbehebung
    \begin{itemize}
      \item Funktionstests
      \item Lasttests
      \item Bugfixing dokumentieren …
    \end{itemize}
\end{enumerate}
\newpage

\chapter*{Literaturverzeichnis}
\addcontentsline{toc}{chapter}{Literaturverzeichnis}
\printbibliography[heading=none]

\end{document}
