\documentclass[12pt,a4paper]{report}

\usepackage[utf8]{inputenc}
\usepackage[T1]{fontenc}
\usepackage[german]{babel}
\usepackage{minted}
\usepackage{csquotes}
\usepackage{graphicx}
\usepackage[colorlinks=true, linkcolor=blue, urlcolor=cyan]{hyperref}
\usepackage[backend=biber,style=authoryear]{biblatex}

\addbibresource{bibliography.bib}

\begin{document}

% ------ Titelblatt ------
\begin{titlepage}
    \centering
    {\Large Hochschule für Musik Karlsruhe \\[1em]}
    {\large Institut für Musikinformatik und Musikwissenschaft \\[6em]}

    {\Large \textbf{Bachelorarbeit} \\[2em]}

    {\LARGE \textbf{Neuer Medienservers für die Bibliothek der Hochschule für Musik Karlsruhe} \\[6em]}

    \begin{minipage}{0.9\textwidth}
        \raggedright
        \textbf{Autor:} Lennart Rathgeb \\
        \textbf{Matrikelnummer:} 13883 \\
        \textbf{Adresse:} Insterburger Straße 2, 76139 Karlsruhe \\
        \textbf{Studiengang:} Musikinformatik (HF), Musikwissenschaft (EF) \\
        \textbf{Erstleser:} Prof. Dr. Christoph Seibert \\
        \textbf{Zweitleser:} Daniel Höpfner \\
    \end{minipage}

    \vfill
    Karlsruhe, den \today
\end{titlepage}

% ------ Inhaltsverzeichnis ------
\cleardoublepage
\pagenumbering{roman}
\tableofcontents

% ------ Abbildungsverzeichnis ------
\cleardoublepage
\listoffigures
\addcontentsline{toc}{chapter}{Abbildungsverzeichnis}
\listoftables
\addcontentsline{toc}{chapter}{Tabellenverzeichnis}

% ------ Einleitung ------
\cleardoublepage
\pagenumbering{arabic}
\setcounter{page}{1}
\chapter*{Einleitung}
\addcontentsline{toc}{chapter}{Einleitung}
\setcounter{section}{0}
\renewcommand\thesection{\arabic{section}}

In der Vergangenheit gab es bereits Projekte einen Medienserver für die Bibliothek der Hochschule für Musik Karlsruhe aufzusetzen. 
Diese Projekte wurden jedoch nie abgeschlossen oder die Systeme wurden nicht dauerhaft betrieben.
Es soll einen ersten Medienserver gegeben haben, jedoch nur mit limitiererter Funktionalität, 
sodass Frederik Schroff im Jahr 2015 als Masterarbeit einen verbesserten Medienserver inklusive Datenbank aufsetzte.
Dieser ist allerdings seit einigen Jahren nicht mehr in Betrieb.
Während meiner Arbeit als Tutor in der Bibliothek, war meine offizielle Aufgabe stehts nur CDs einzulesen, zu benennen und abzuspeichern.
Diese waren anschließend aber nicht abrufbar, sodass ich anfing die Arbeit zu hinterfragen.
Nachdem ich als Praxisarbeit bereits den Digitalisierungsprozess der CDs durch die Erneuerung des dafür verwendeten Skripts verbesserte, 
möchte ich nun den Medienserver neu aufsetzen, damit die angesammelten Daten auch genutzt werden können.

Ziel ist es den Server so zu gestalten, dass er einfach zu betreiben ist und über eine benutzerfreundliche Oberfläche verfügt.
Es soll ermöglicht werden, die digitalisierten CDs damit unter Beachtung rechtlicher Rahmenbedingungen zu streamen.

\section{Zur Dokumentation}
Die Dokumentation Gliedert sich in zwei Teile, wobei der erste Teil die technischen Hintergründe beschreibt
und der zweite Teil auf die praktische Umsetzung in der Bibliothek eingeht.
Damit dient sie nicht nur dazu einen Einstieg in den Themenbereich Netzwerk und Serveradminastration zu geben, 
sondert bildet auch die Anleitung zur Nutzung und Wartung des Medienservers in der Bibliothek ab.

Im ersten Teil werden grundlegende Begriffe erläutert, 
die Entwicklung von Servern und Medienservern nachgezeichnet 
sowie deren Anwendungsbereiche dargestellt. 
Darauf aufbauend werden technische Aspekte der Serverarchitektur, 
der Administration und der Netzwerktechnik vorgestellt, 
bevor auf spezifische Eigenschaften von Medienservern eingegangen wird. 
Abschließend wird die besondere Rolle von Medienservern im 
Bibliothekskontext betrachtet. 

Der zweite Teil beschreibt die praktische Umsetzung des Projekts. 
Hier wird zunächst die Zielsetzung definiert und die notwendigen 
Voraussetzungen hinsichtlich Hard- und Software sowie des Netzwerks erläutert. 
Es folgt die detaillierte Installation und Konfiguration der 
benötigten Komponenten, die Integration in die bestehende IT-Infrastruktur 
sowie die Einrichtung der Benutzerverwaltung und Zugriffsrechte. 
Darüber hinaus werden Testverfahren und Strategien zur 
Fehlerbehebung dokumentiert. 

Den Abschluss der Arbeit bilden ein Fazit mit einer 
kritischen Reflexion des Projekts sowie ein Ausblick 
auf mögliche Weiterentwicklungen.

% ------ Theoretische Grundlagen ------
\chapter*{Theoretische Grundlagen}
\addcontentsline{toc}{chapter}{Theoretische Grundlagen}
\setcounter{section}{0}

\section{Einführung in Server und Medienserver}
Server bilden das Fundament moderner IT-Infrastrukturen.
Sie übernehmen zentrale Aufgaben bei der Bereitstellung von Diensten 
und der Verwaltung von Daten. 
Im Zusammenhang mit Medienservern steht vor allem die Speicherung, 
Organisation und Übertragung digitaler Inhalte wie Musik, 
Videos oder Bilder im Vordergrund. 

Medienserver ermöglichen es, solche Inhalte in Echtzeit zu streamen, 
ohne dass ein vollständiger Download notwendig ist. 
Ihre Entwicklung ist eng mit der allgemeinen Geschichte von Servertechnologien und Netzwerken verknüpft 
und reicht von klassischen Rechenzentren über den Aufstieg des Internets bis hin zu modernen Heim- und Cloudlösungen. 

Heute finden Medienserver Einsatz in unterschiedlichsten Bereichen - 
von der privaten Nutzung über Unternehmen bis hin zu Bibliotheken, 
die digitalisierte Bestände für ihre Nutzer zugänglich machen.

  \subsection{Begriffsdefinitionen}
  Als \emph{Server} bezeichnet man einen „Rechner, 
  der für andere in einem Netzwerk mit ihm verbundene Systeme bestimmte Aufgaben übernimmt 
  und von dem diese ganz oder teilweise abhängig sind“. \footcite{duden_server}

  In diesem Zusammenhang ist ein \emph{Netzwerk} ein Zusammenschluss von „zwei oder mehr Computern,
  die miteinander verbunden sind, um Daten elektronisch auszutauschen.“
  (Übersetzung nach \footcite{britannica_network})

  Ein \emph{Medienserver} ist eine spezialisierte Form eines Servers, 
  die darauf ausgelegt ist, digitale Medieninhalte wie Audio, Video oder Bilder zu speichern, 
  zu verwalten und über ein Netzwerk für Clients bereitzustellen.\footcite[Vgl.][S.~131 ~ff.]{steinmetz_multimedia}

  Unter \emph{Streaming} versteht man die Übertragung von Audio- bzw. Videodaten von einem Server auf einen Client, 
  wobei die Wiedergabe umgehend erfolgt.\footcite{gabler_streaming}
  Multimedia-Inhalten können so bereits während der Übertragung konsumiert werden, 
  ohne dass eine vollständige Speicherung der Daten auf dem Endgerät erforderlich ist.

  Server und Clients bilden damit die grundlegende Kommunikationsbasis in Netzwerken. 
  Medienserver stellen eine spezielle Ausprägung dar, 
  die insbesondere für die effiziente Bereitstellung von Streaming-Inhalten optimiert sind.

  \subsection{Historische Entwicklung}
  Die Entwicklung von Servern und Netzwerken ist eng mit der allgemeinen Geschichte der Computerkommunikation verknüpft. 
  Erste Formen von Rechnernetzen entstanden bereits Ende der 1950er Jahre mit dem ARPANET, 
  das als Vorläufer des heutigen Internets gilt 
  und den Grundstein für die paketorientierte Datenübertragung legte.\footcite[Vgl.][S.~45~f.]{tanenbaum_computernetworks}

  In den frühen 1970er Jahren zeigte sich, dass die bestehenden Protokolle des ARPANET nicht ausreichten, 
  um heterogene Netzwerke wie Funk- oder Satellitennetze zu integrieren. 
  Vor diesem Hintergrund entwickelten Vinton Cerf und Robert Kahn 1974 das Transmission Control Protocol (TCP), 
  das später in die heute gebräuchliche Protokollfamilie TCP/IP aufgeteilt wurde. 
  TCP/IP ermöglichte die zuverlässige Kommunikation über unterschiedliche Netzwerke hinweg 
  und bildete die Grundlage für die Interoperabilität moderner Netze.\footcite[Vgl.][S.~45]{tanenbaum_computernetworks}
  Im Jahr 1983 wurde TCP/IP im ARPANET als verbindlicher Standard eingeführt,
  was gemeinhin als Geburtsstunde des modernen Internets gilt.\footnote{Vgl. \cite[S.~2]{postel_rfc801}; \cite{britannica_tcpip}.}

  In den 1980er und 1990er Jahren etablierte sich das Client-Server-Modell als dominierendes Paradigma für den Datenaustausch. 
  Hierbei übernehmen Server die Rolle zentraler Dienstleister, 
  während Clients auf deren Ressourcen zugreifen.\footcite[Vgl.][S.~114]{kurose_networking} 
  Mit der zunehmenden Verbreitung des Internets wurden Server in Unternehmen, 
  Hochschulen und öffentlichen Institutionen zu zentralen Infrastrukturen.

  Parallel dazu entwickelte sich der Bedarf an multimedialer Datenübertragung. 
  Erste Streaming-Verfahren wie RealAudio (1995) ermöglichten die kontinuierliche Übertragung von Audioinhalten, 
  ohne dass die Daten vollständig gespeichert werden mussten. 
  Diese Technologie wurde später auf Video übertragen 
  und in standardisierte Protokolle wie Real-time Transport Protocol (RTP) überführt.\footcite[Vgl.][S.~11~ff., S.~273~ff.]{steinmetz_multimedia}

  Seit den 2000er Jahren prägen Breitbandinternet und die Verbreitung mobiler Endgeräte die Nutzung multimedialer Inhalte. 
  Parallel dazu entwickelten sich adaptive Streaming-Technologien wie MPEG-DASH (Standardisierung 2012)
  und Apple HLS (2009), die eine effiziente und bandbreitenabhängige Übertragung von Audio- 
  und Videoinhalten ermöglichen.\footcite[Vgl.]{iso_mpegdash, apple_hls}
  Medienserver sind in diesem Kontext zu zentralen Plattformen für die Speicherung und Bereitstellung digitaler Bestände geworden 
  und finden heute sowohl im privaten Umfeld als auch in Institutionen wie Hochschulen und Bibliotheken Anwendung.

  \subsection{Anwenwendungsbereiche}
  Server übernehmen zentrale Aufgaben in der IT, etwa als Datei-, Datenbank-, Web- oder E-Mail-Server. 
  Darüber hinaus kommen sie in Bereichen wie Virtualisierung, Cloud-Computing
  oder bei Sicherheitsdiensten (z.\,B. Authentifizierung, Firewalls) zum Einsatz.

  Medienserver bilden eine spezialisierte Kategorie: Sie speichern und verwalten
  digitale Audio-, Video- und Bildbestände und stellen diese über Netzwerke bereit.
  Einsatzfelder finden sich im privaten Umfeld (Streaming von Musik und Filmen),
  in Unternehmen (z.\,B. interne Schulungsvideos oder Marketingmaterial),
  im Bildungsbereich (Vorlesungsaufzeichnungen, E-Learning-Plattformen) sowie
  in Kulturinstitutionen wie Archiven und Bibliotheken, wo sie die Nutzung und
  Langzeitverfügbarkeit digitalisierter Sammlungen unterstützen.\footcite[Vgl.][S.~131~ff.]{steinmetz_multimedia}

\section{Grundlagen der Serverarchitektur}
Serverarchitekturen umfassen sowohl Hardware- als auch Softwareaspekte. 
Sie bilden die Grundlage für den Betrieb spezialisierter Systeme wie Medienserver, 
die neben klassischen Serverkomponenten auch spezielle Anforderungen an Speicher und Datenübertragung stellen.

  \subsection{Hardware- und Softwareaspekte}
  Ein Server besteht aus zentralen Hardwarekomponenten wie Prozessor, Hauptspeicher, Netzwerk- und Massenspeicherhardware. 
  Für den Dauerbetrieb sind Aspekte wie Zuverlässigkeit, Ausfallsicherheit, Energieeffizienz und Skalierbarkeit besonders wichtig. 
  Anders als bei Arbeitsplatzrechnern kommen häufig redundante Netzteile, unterbrechungsfreie Stromversorgungen, 
  spezielle Kühlungssysteme sowie redundante Netzwerkverbindungen zum Einsatz, 
  um eine kontinuierliche Verfügbarkeit zu gewährleisten.\footcite[Vgl.][S.~152541]{ahmed2021energy} 

  % Auch die Netzwerkanbindung ist von zentraler Bedeutung: 
  % Mehrport-Netzwerkkarten, redundante Switch-Anbindungen oder Load-Balancing-Verfahren 
  % stellen sicher, dass ein Server auch bei steigender Last oder im Fehlerfall 
  % funktionsfähig bleibt.\footcite[Vgl.][?]{tanenbaum_computernetworks}

  Auf der Softwareseite bilden Betriebssysteme die Grundlage für den Betrieb, 
  wobei sich stabile, sichere und gut wartbare Plattformen durchgesetzt haben. 
  Darüber hinaus sind Serverdienste (z.\,B. Web-, Datenbank- oder Streamingdienste) 
  sowie Middleware-Komponenten erforderlich, die den Zugriff von Clients ermöglichen 
  und die Kommunikation zwischen verschiedenen Anwendungen koordinieren. 
  Gerade bei Medienservern sind zudem Softwarelösungen zur effizienten Verwaltung, 
  Katalogisierung und Bereitstellung multimedialer Inhalte notwendig.

  \subsection{Betriebssysteme für Server}
  Zu den gängigsten Server-Betriebssystemen zählen insbesondere 
  Linux-Distributionen, Windows Server sowie UNIX-Derivate.
  Linux gilt als besonders verbreitet im Bereich Web- und Medienserver, 
  da es durch Stabilität, Sicherheit und Flexibilität überzeugt und 
  vor allem als bevorzugte Plattform für Webserver wie Apache dient.\footcite[Vgl.][S.~963~ff.]{nemeth_unixlinux} 
  Windows Server findet häufig Anwendung in Unternehmensumgebungen, 
  insbesondere durch die Integration mit Active Directory, 
  das als zentrales Verzeichnis- und Authentifizierungssystem eine 
  wichtige Rolle spielt.\footcite[Vgl.][S.~1154~f.]{nemeth_unixlinux}

  \subsection{Virtualisierung und Containerisierung}
  Moderne Serverarchitekturen setzen zunehmend auf Virtualisierung und Containerisierung, 
  um Hardware effizienter auszunutzen und Dienste flexibel bereitzustellen. 

  Klassische Virtualisierungslösungen wie VMware oder KVM ermöglichen die parallele Ausführung mehrerer 
  virtueller Maschinen auf einer physischen Hardware.\footcite[Vgl.][S.~1005, S.~995~f.]{nemeth_unixlinux} 

  Containerisierungstechnologien wie Docker und Kubernetes gehen darüber hinaus einen Schritt weiter, 
  indem sie Anwendungen in isolierten Umgebungen ausführen und dabei besonders ressourcenschonend sowie 
  skalierbar gestalten. Wie im Beitrag von Mao et al. (2020) hervorgehoben wird, 
  bildet „die container-basierte Virtualisierungstechnologie, 
  wie Docker und Kubernetes, die Grundlage für cloud-native Anwendungen“.\footcite{mao2020containers}

\subsection{Dateisysteme und Speicherlösungen}
Für Medienserver spielt die Wahl der Speicherarchitektur eine zentrale Rolle. 
Netzwerkspeicherlösungen wie Network Attached Storage (NAS) und Storage Area Networks (SAN) 
ermöglichen die zentrale Bereitstellung großer Datenmengen. 
NAS-Systeme stellen Speicher im Netzwerk auf Dateiebene bereit und zeichnen sich durch einfache Einrichtung, 
schnellen Datenzugriff und unkomplizierte Administration aus.\footcite[Vgl.][S.~21857]{saravanamuthu2014study} 
SANs hingegen arbeiten auf Blockebene und sind für besonders leistungsintensive Anwendungen optimiert; 
sie bieten hohe Bandbreiten, niedrige Latenzen und ermöglichen die zentrale Verwaltung großer Speicherpools.\footcite[Vgl.][S.~21858~f.]{saravanamuthu2014study}  

Moderne Speicherlösungen sind darüber hinaus auf niedrige Latenz, hohe Bandbreite und Skalierbarkeit ausgelegt. 
In experimentellen Vergleichen konnte gezeigt werden, dass SAN-Systeme bei großen Datenmengen und intensiven Workflows deutliche Leistungs- 
und Latenzvorteile gegenüber NAS haben.\footcite[Vgl.][Abschnitt~IV; Ergebnisse]{jaikar2016performance}  
NAS-Systeme dagegen bieten Dateizugriff über verbreitete Protokolle (z. B. NFS, SMB) 
und sind häufig kostengünstiger in Anschaffung und Wartung.\footcite[Vgl.][S.~21857]{saravanamuthu2014study}


Neben der Netzwerkanbindung ist Redundanz ein entscheidender Faktor: Mechanismen 
wie RAID (Redundant Array of Independent Disks) schützen vor Datenverlust durch 
Festplattenausfall, während Hot-Swapping und automatische Neusynchronisation 
die Verfügbarkeit auch im Fehlerfall erhöhen. In der klassischen Untersuchung „RAID: 
High-Performance, Reliable Secondary Storage“ wird herausgearbeitet, dass RAID-Systeme durch Striping über mehrere Festplatten 
und durch Redundanz eine signifikante Verbesserung von Leistung und Zuverlässigkeit bieten, insbesondere bei großen Datenmengen 
und paralleler Zugriffslast.\footcite[Vgl.][151 ~ff.]{chen2004raid}

\section{Serveradministration}
Die Administration eines Servers umfasst grundlegende Prozesse der Benutzerverwaltung, Wartung und Absicherung. 
Sie bildet die Voraussetzung für einen stabilen und sicheren Betrieb und ist in der Praxis oft ein kontinuierlicher Prozess. % \footcite[Vgl.][Kapitel~1.1]{nemeth_unixlinux}

  \subsection{Benutzer- und Rechteverwaltung}
  Ein zentrales Element der Serveradministration ist die Verwaltung von Benutzern und deren Zugriffsrechten. 
  In UNIX- und Linux-Systemen erfolgt die Identifikation über eindeutige Benutzerkennungen (User IDs, UIDs) 
  und Gruppenzugehörigkeiten (Group IDs, GIDs).\footcite[Vgl.][S.~176~ff.]{nemeth_unixlinux}
  Das klassische UNIX-Berechtigungsmodell unterscheidet dabei drei grundlegende Kategorien: 
  \emph{user} (Besitzer), \emph{group} (Mitglieder der zugewiesenen Gruppe) und \emph{others} (alle übrigen Nutzer). 
  Für jede dieser Kategorien lassen sich Lese- (read, r), Schreib- (write, w) und Ausführungsrechte (execute, x) festlegen.\footcite[Vgl.]{wikipedia_dateisystemrechte}

  Zentral ist zudem die Trennung zwischen normalen Benutzerkonten und dem privilegierten \emph{Superuser} (\texttt{root}), 
  der uneingeschränkten Zugriff auf das gesamte System besitzt. 
  Viele moderne Distributionen nutzen anstelle direkter Root-Anmeldungen das \texttt{sudo}-Kommando, 
  das einzelnen Benutzern temporär administrative Rechte verleiht und gleichzeitig Aktionen protokolliert.\footcite[Vgl.][S.~112~ff.]{nemeth_unixlinux}

  Erweiterte Mechanismen wie \emph{Access Control Lists (ACLs)} ermöglichen eine feinere Abstufung der Berechtigungen, 
  indem sie Zugriffsrechte für mehrere Benutzer und Gruppen individuell definieren.\footcite[Vgl.][S.~159~ff.]{nemeth_unixlinux} 
  In einigen Systemen wie Solaris findet darüber hinaus eine rollenbasierte Zugriffskontrolle (\emph{Role-Based Access Control, RBAC}) Anwendung. 
  Dieses Konzept weist Rechte nicht mehr einzelnen Benutzern, sondern Rollen zu, die wiederum von Benutzern übernommen werden können. 
  Dadurch lassen sich administrative Aufgaben gezielt delegieren, ohne umfassende Superuser-Rechte vergeben zu müssen.\footcite[Vgl.][S.~108]{nemeth_unixlinux}

  \subsection{Dienste und Prozesse}
  Die Funktionsfähigkeit eines Servers basiert wesentlich auf Prozessen und Diensten. 
  Während Prozesse allgemein laufende Programme im Betriebssystem darstellen, 
  sind Dienste spezielle Hintergrundprozesse, die in UNIX- und Linux-Systemen 
  traditionell als \emph{daemons} bezeichnet werden.\footcite[Vgl.]{wikipedia_daemon}  

  Die Verwaltung dieser Dienste erfolgt über das \emph{init}-System, 
  das beim Bootvorgang automatisch Prozesse startet und kontrolliert. 
  Dabei kommen distributionsspezifische Startskripte zum Einsatz.\footcite[Vgl.][S.~88~ff.]{nemeth_unixlinux}  

  In vielen modernen Linux-Distributionen hat sich \emph{systemd} etabliert, 
  das Dienste parallelisiert starten kann und erweiterte Funktionen wie Abhängigkeitsmanagement 
  sowie integriertes Logging bietet.\footcite[Vgl.]{wikipedia_systemd}   

  \subsection{Monitoring und Logging}
  Für den Betrieb von Servern ist die kontinuierliche Überwachung essenziell. 
  Monitoring umfasst sowohl die Ressourcennutzung (CPU, RAM, Festplatten, Netzwerk) als auch die Erreichbarkeit zentraler Dienste. 
  Hierfür stehen je nach System Werkzeuge wie \texttt{top} (Linux), \texttt{topas} (AIX) oder \texttt{prstat} (Solaris) zur Verfügung, 
  die eine regelmäßig aktualisierte Übersicht über aktive Prozesse und deren Ressourcennutzung bieten.\footcite[Vgl.][S.~133]{nemeth_unixlinux}
  Darüber hinaus ermöglichen sie Administratoren auch Eingriffe, 
  etwa durch das Senden von Signalen oder das Anpassen der Priorität von Prozessen.\footcite[Vgl.][S.~134]{nemeth_unixlinux}

  Logging erfolgt primär über das \emph{Syslog}-Framework, das Ereignisse zentral erfasst und in Logdateien strukturiert speichert.\footcite[Vgl.][S.~344~ff.]{nemeth_unixlinux}  
  Eine strukturierte Protokollauswertung ist entscheidend, um Fehlerquellen und sicherheitsrelevante Vorfälle nachvollziehen zu können. 
  Moderne Systeme ermöglichen zudem die Weiterleitung von Logs an zentrale Server oder die Integration in Analyseplattformen, 
  um eine konsolidierte Überwachung zu gewährleisten.\footcite[Vgl.][S.~348]{nemeth_unixlinux}

  \subsection{Backup- und Recovery-Strategien}
  Datenverluste durch Hardwareausfälle oder Benutzerfehler gehören zu den größten Risiken im Serverbetrieb. 
  Daher sind regelmäßige Backups sowie erprobte Wiederherstellungsstrategien unverzichtbar. 
  Gängige Verfahren sind \emph{inkrementelle} und \emph{differenzielle Backups}. 
  Bei inkrementellen Sicherungen werden jeweils nur die seit der letzten Sicherung 
  geänderten Daten gespeichert, während differenzielle Backups stets alle Änderungen 
  seit der letzten Vollsicherung erfassen.\footcite[Vgl.][S.~305~f.]{nemeth_unixlinux}  
  Diese Methoden reduzieren den Aufwand und Speicherbedarf gegenüber Vollsicherungen erheblich. 

  Für kritische Systeme werden zusätzlich \emph{Offsite-Backups} eingesetzt. 
  Das bedeutet, dass Kopien der Daten an einem geographisch getrennten Standort gespeichert werden, 
  sei es durch das Versenden physischer Medien oder durch Übertragung an entfernte Speicher/Cloud-Dienste. 
  Diese Maßnahme schützt vor lokalen Katastrophen wie Brand oder Naturereignissen 
  und ermöglicht eine schnelle Wiederaufnahme des Betriebs, wenn die Hauptanlage nicht nutzbar ist.\footcite[Vgl.]{wikipedia_offsite_data_protection}  

  \subsection{Sicherheitsaspekte}
  Die Absicherung von Servern erfordert ein mehrschichtiges Konzept, 
  das organisatorische, technische und prozessuale Maßnahmen umfasst.
  Ein ganzheitliches Sicherheitskonzept, das möglichst viele der folgenden grundlegenden Aspekte enthält ist unverzichtbar, 
  um Server vor Angriffen und Ausfällen zu schützen und die Verfügbarkeit kritischer Dienste sicherzustellen.
  \\
  \newline
  \textbf{Netzwerkschutz}:  
  Ein zentraler Bestandteil der Serversicherheit ist der Schutz der Netzwerkschnittstellen. 
  Hierzu werden Firewalls eingesetzt, um eingehenden und ausgehenden Datenverkehr anhand von Regeln zu kontrollieren. 
  Klassische Paketfilter analysieren Header-Informationen wie Quell- und Zieladresse oder Ports und können so unerwünschte Verbindungen blockieren.\footcite[Vgl.][S.~932~ff.]{nemeth_unixlinux}  
  \\
  \newline
  \textbf{Verschlüsselung}:  
  Um die Vertraulichkeit und Integrität der Kommunikation zu gewährleisten, werden Verschlüsselungstechniken eingesetzt. 
  Transport Layer Security (TLS/SSL) schützt Datenströme im Internet, Virtual Private Networks (VPNs) verschlüsseln den gesamten Datenverkehr zwischen Standorten, 
  und Secure Shell (SSH) ermöglicht eine abgesicherte Fernadministration.\footcite[Vgl.][S.~801, S.~971, S.~926~ff., S.~942]{nemeth_unixlinux}  
  \\
  \newline
  \textbf{Benutzer- und Rechteverwaltung}:  
  Ein konsequentes Berechtigungsmanagement stellt sicher, dass Benutzer nur Zugriff auf die Ressourcen haben, die sie tatsächlich benötigen. 
  Dazu gehören sichere Passwörter, Mechanismen wie \emph{Password Aging} (zeitliche Begrenzung der Passwortgültigkeit) 
  und die Vergabe von Rechten nach dem Prinzip des geringsten Privilegs. 
  So wird verhindert, dass kompromittierte Accounts zu weitreichenden Sicherheitsproblemen führen.\footcite[Vgl.][S.~905~ff.]{nemeth_unixlinux}  
  \\
  \newline
  \textbf{Systemhärtung}:  
  Viele Systeme sind in der Standardkonfiguration mit unnötigen Diensten ausgestattet, die zusätzliche Angriffsflächen bieten. 
  Systemhärtung bedeutet, diese Dienste konsequent zu deaktivieren und Konfigurationen sicherheitsorientiert vorzunehmen. 
  Damit sinkt die Wahrscheinlichkeit, dass Angreifer über Fehlkonfigurationen oder selten genutzte Software ins System eindringen können.\footcite[Vgl.][S.~902]{nemeth_unixlinux}
  \\  
  \newline
  \textbf{Updates und Patching}:  
  Die meisten erfolgreichen Angriffe erfolgen über bekannte, aber nicht geschlossene Schwachstellen. 
  Daher ist ein zeitnahes Einspielen von Patches für Betriebssystem und Anwendungssoftware unerlässlich. 
  Ein strukturierter Patch-Management-Prozess reduziert das Risiko signifikant.\footcite[Vgl.][S.~901]{nemeth_unixlinux}  
  \\
  \newline
  \textbf{Intrusion Detection und Monitoring}:  
  Intrusion Detection Systeme (IDS) überwachen den Datenverkehr oder Host-Systeme auf verdächtige Aktivitäten. 
  Intrusion Prevention Systeme (IPS) gehen noch einen Schritt weiter und können erkannte Angriffe automatisch blockieren. 
  Beispiele sind Snort (Netzwerk-IDS) oder OSSEC (Host-basiertes IDS). 
  In Kombination mit einer kontinuierlichen Log-Analyse ermöglichen diese Werkzeuge eine frühzeitige Erkennung von Angriffen.\footcite[Vgl.][S.~918~ff.]{nemeth_unixlinux} 
  \\ 
  \newline
  \textbf{Backups und Recovery}:  
  Auch wenn alle präventiven Maßnahmen ergriffen werden, lassen sich Vorfälle nicht vollständig verhindern. 
  Regelmäßige Backups sind daher ein essenzielles Element der Sicherheit, da sie eine Wiederherstellung nach Angriffen oder Ausfällen ermöglichen. 
  Besonders im Kontext von Ransomware oder Datenmanipulation sind getestete Recovery-Strategien unverzichtbar.\footcite[Vgl.][S.~903]{nemeth_unixlinux} 
  \\ 
  \newline
  \textbf{Proaktives Sicherheitsmanagement}:  
  Technische Maßnahmen reichen allein nicht aus. 
  Ein umfassendes Sicherheitskonzept verlangt nach klar definierten Richtlinien, kontinuierlichen Audits und einem hohen Maß an Wachsamkeit. 
  Administratoren sollten regelmäßig die Wirksamkeit der Maßnahmen prüfen und neue Bedrohungslagen aktiv in die Planung einbeziehen. 
  So wird Sicherheit zu einem fortlaufenden Prozess und nicht zu einem einmaligen Projekt.\footcite[Vgl.][S.~901, S.~905]{nemeth_unixlinux}  

\section{Netzwerktechnik für Medienserver}
Netzwerke bilden die Grundlage für den Zugriff auf Medienserver. 
Dieser Abschnitt erläutert zentrale Konzepte, Protokolle sowie Leistungs- und Sicherheitsaspekte, 
die für den effizienten Betrieb relevant sind.  

  \subsection{Grundlagen der Netzwerktechnik}  
  Die Kommunikation in Netzwerken basiert auf dem \emph{TCP/IP}-Protokollstapel. 
  \emph{IP} (\emph{Internet Protocol}) ist für die Adressierung und Weiterleitung von Paketen zuständig 
  und bildet die Grundlage für die Kommunikation in paketvermittelten Netzwerken.\footcite[Vgl.][S.~438]{tanenbaum_computernetworks}  
  Darüber hinaus sorgen \emph{TCP} (\emph{Transmission Control Protocol}) und \emph{UDP} (\emph{User Datagram Protocol}) 
  auf der Transportschicht für den eigentlichen Datentransport. 
  TCP stellt eine verbindungsorientierte, zuverlässige Übertragung mit Fehlerkorrektur und Flusskontrolle bereit, 
  während UDP ein verbindungsloses, leichtgewichtiges Protokoll ist, 
  das insbesondere für zeitkritische Anwendungen wie Audio- und Videostreaming eingesetzt wird.\footcite[Vgl.][S.~228]{kurose_networking}
 
  Dienste werden über \emph{Ports} identifiziert, etwa Port~80 für HTTP oder Port~443 für HTTPS.\footcite[Vgl.][S.~554]{tanenbaum_computernetworks}  
  Für die Namensauflösung sorgt das Domain Name System (DNS), 
  das menschenlesbare Hostnamen in IP-Adressen übersetzt 
  und somit die Grundlage für jede netzwerkbasierte Medienübertragung bildet.\footcite[Vgl.][S.~611~f.]{tanenbaum_computernetworks}  

  \subsection{Streaming-Protokolle} 
  Zur Übertragung von Multimedia-Inhalten existieren verschiedene Protokolle. 
  Die wichtigesten werden im Folgenden vorgestellt.  
  \\
  \newline
  \textbf{HTTP (Hypertext Transfer Protocol):}
  HTTP ist ein Anwendungsprotokoll auf der Anwendungsschicht und bildet das Herzstück des World Wide Web. 
  Es wird in einem Client- und einem Server-Programm implementiert, die über den Austausch von HTTP-Nachrichten kommunizieren. 
  Diese Nachrichten sind nach einem standardisierten Format (RFC 1945, RFC 2616) definiert und regeln die Interaktion zwischen Webbrowser und Webserver.\footcite[Vgl.][S.~126]{kurose_networking} 
  HTTP ist dabei ein \emph{zustandsloses} Protokoll, was bedeutet, dass der Server keine Informationen über vorherige Anfragen speichert. 
  Jede Anfrage wird unabhängig behandelt, wodurch HTTP besonders einfach und skalierbar ist.\footcite[Vgl.][S.~128]{kurose_networking}   
  \\
  \newline
  \textbf{RTP (Real-Time Transport Protocol):}  
  RTP transportiert Audio- und Videodaten meist über \emph{UDP (User Datagram Protocol)}. 
  Es ist speziell für Echtzeitanwendungen optimiert und unterstützt Mechanismen zur 
  Synchronisation verschiedener Medienstöme (z.\,B. Audio und Video) 
  sowie zur Kompensation von Jitter, der durch schwankende Paketlaufzeiten entsteht.\footcite[Vgl.][S.~546~ff.]{tanenbaum_computernetworks}   
  \\
  \newline
  \textbf{RTSP (Real-Time Streaming Protocol):}  
  RTSP ist ein Steuerungsprotokoll, das Befehle wie \emph{Play}, \emph{Pause} oder \emph{Seek} ermöglicht. 
  Es wird häufig zusammen mit dem \textbf{RTP (Real-Time Transport Protocol)} verwendet, 
  das die eigentliche Datenübertragung übernimmt.\footcite[Vgl.][S.~638~f.]{kurose_networking}  
  \\
  \newline
  \textbf{HLS (HTTP Live Streaming):}
  HLS ist ein von Apple entwickeltes Streaming-Protokoll, das in \emph{RFC 8216} standardisiert ist.
  Es basiert auf der Idee, Medieninhalte in kleine Segmente (Chunks) zu zerlegen, die sequenziell über HTTP übertragen werden.
  Ein Manifest (Playlist) beschreibt die Abfolge dieser Segmente, wodurch auch unterschiedliche Qualitätsstufen bereitgestellt werden können.
  Dies ermöglicht \emph{adaptives Streaming}, bei dem der Client dynamisch zwischen verschiedenen Bitraten und Auflösungen wechseln kann,
  um Schwankungen in der Netzwerkbandbreite auszugleichen.\footcite[Vgl.][siehe HTTP Live Streaming Overview]{rfc8216}
  Dank der Nutzung von HTTP ist HLS besonders kompatibel mit bestehenden Infrastrukturen wie Caches und Content Delivery Networks (CDNs).\footcite[Vgl.][]{stockhammer2011dash}
  \\
  \newline
  \textbf{MPEG-DASH (Dynamic Adaptive Streaming over HTTP):}
  MPEG-DASH ist ein von der MPEG-Gruppe entwickelter Standard für adaptives Multimedia-Streaming über HTTP.
  Das Grundprinzip besteht darin, dass Multimedia-Inhalte in kleine Segmente zerlegt werden, die über herkömmliche HTTP-Server bereitgestellt werden.
  Der Client entscheidet dynamisch, welche Segmente er anfordert, um sich an schwankende Netzwerkbedingungen und Geräteeigenschaften anzupassen.\footcite[Vgl.][S.~1~ f.]{stockhammer2011dash}
  Dadurch kombiniert DASH die weite Verbreitung von HTTP-Infrastrukturen mit der Möglichkeit eines qualitativ stabilen Medienkonsums auch unter variablen Bandbreitenbedingungen.
    
  \subsection{Leistungsanforderungen}  
  Streaming-Anwendungen stellen hohe Anforderungen an die Netzwerkinfrastruktur. 
  Neben ausreichender \textbf{Bandbreite} ist vor allem geringe \textbf{Latenz} entscheidend, 
  insbesondere bei Live-Übertragungen oder interaktiven Anwendungen. 
  Eine wesentliche Rolle spielt außerdem die \textbf{Pufferung}, 
  die Jitter (Schwankungen in der Paketlaufzeit) ausgleichen soll. 
  Typische Verfahren sind Pre-Buffering und adaptive Puffergrößen.\footcite[Vgl.][S.~697~ff.]{tanenbaum_computernetworks}  

  \subsection{Qualität und Sicherheit}  
  Die Sicherstellung einer konstanten Übertragungsqualität erfordert \textbf{Quality of Service (QoS)}-Mechanismen.  
  QoS bezeichnet Techniken, die Netzwerkressourcen gezielt steuern, um die Anforderungen von Multimedia-Anwendungen zu erfüllen.\footcite[Vgl.][S.~9~ff.]{steinmetz_multimedia}  
  Im Folgenden werden einige gängige QoS-Mechanismen vorgestellt: 
  \\
  \newline
  \textbf{Traffic Shaping:}  
  Verfahren wie \emph{Token Bucket} oder \emph{Leaky Bucket} regulieren die Burstartigkeit von Datenströmen, indem sie den Datenfluss glätten und Spitzenlasten abfangen.  
  Dadurch wird ein gleichmäßigerer Durchsatz erreicht, der insbesondere für kontinuierliche Audio- und Videoübertragungen wichtig ist.\footcite[Vgl.][S.~61~f.]{steinmetz_multimedia}  
  \\
  \newline
  \textbf{Rate Control und Scheduling:}  
  Kontrollmechanismen wie \emph{Fair Queueing}, \emph{Virtual Clock} oder \emph{Delay Earliest-Due-Date} sorgen dafür, 
  dass Bandbreiten fair aufgeteilt und Verzögerungen minimiert werden.  
  Zugleich helfen sie, Jitter auszugleichen und zeitkritische Datenströme zu priorisieren.\footcite[Vgl.][S.~62~f.]{steinmetz_multimedia}  
  \\
  \newline
  \textbf{Resource Reservation:}  
  Um Paketverluste und Verzögerungen zu vermeiden, können Bandbreiten, Puffer und andere Ressourcen im Voraus reserviert werden.  
  Dies kann pessimistisch (Worst-Case) oder optimistisch (Durchschnittswert) erfolgen und garantiert so Mindestqualitäten für Multimediadienste.\footcite[Vgl.][S.~52~f.]{steinmetz_multimedia}  
  \\
  \newline
  \textbf{Admission Control:}  
  Bevor eine neue Verbindung zugelassen wird, prüft das Netzwerk, ob genügend Ressourcen vorhanden sind.  
  Fehlen diese, wird die Anfrage abgelehnt, um eine Überlastung zu verhindern und die Qualität bestehender Verbindungen nicht zu beeinträchtigen.\footcite[Vgl.][S.~50~f.]{steinmetz_multimedia}  
  \\
  \newline
  \textbf{Error Control:}  
  Fehlererkennungs- und Fehlerkorrekturverfahren (z.\,B. FEC - Forward Error Correction) sind essenziell, 
  um trotz Paketverlusten eine akzeptable Qualität sicherzustellen.  
  Dabei können Redundanzdaten eingesetzt werden, die eine Rekonstruktion verlorener Informationen ermöglichen.\footcite[Vgl.][S.~68]{steinmetz_multimedia}  
  \\
  \newline  
  Aspekte der Sicherheit, die beim Betrieb eines Medienservers berücksichtigt werden müssen,
  wurden im Abschnitt „Serveradministration“ unter „Sicherheitsaspekte“ bereits beschrieben. 
    
\section{Spezifisches zu Medienservern}  
Medienserver unterscheiden sich von klassischen Servern vor allem durch ihre Spezialisierung auf die Verwaltung, Transkodierung und Auslieferung multimedialer Inhalte.  
Sie kombinieren Datenbanken, Streaming-Protokolle und Benutzerverwaltung und müssen dabei sowohl technische als auch rechtliche Anforderungen berücksichtigen.  
  
  \subsection{Typische Medienserver-Software}  
  Es existiert eine Vielzahl an Softwarelösungen, die sich im Funktionsumfang, in der Lizenzierung und in den Einsatzszenarien unterscheiden:  
  \\
  \newline 
  \textbf{Plex:}
  Plex zählt zu den bekanntesten Medienservern und wird vor allem im privaten Umfeld sowie in kleineren Organisationen genutzt, 
  um Audio-, Video- und Bilddateien zentral zu verwalten und auf verschiedensten Endgeräten bereitzustellen. 
  Typische Anwendungsfelder sind Heimnetzwerke, in denen Plex als Medienzentrale für Smart-TVs, Streaming-Boxen, Tablets und Smartphones dient. 
  Ein zentrales Merkmal ist die Echtzeit-Transkodierung, durch die Inhalte unabhängig von Format oder Endgerät konsumierbar werden. 
  Ergänzend bietet Plex mobile Clients, Webinterfaces und Zusatzfunktionen wie Offline-Synchronisation oder Live-TV, 
  die teilweise nur im kostenpflichtigen Abonnement „Plex Pass“ verfügbar sind.\footcite[Vgl.][Abschnitt „Plex Media Server“]{wikipedia_plex}\footcite[Vgl.][]{plex_docs}
  \\
  \newline
  \textbf{Emby:}
  Emby ist eine proprietäre Medienserver-Software, die Endgeräte wie Smart-TVs, Mobilgeräte und Webbrowser unterstützt. 
  In der Grundversion kostenlos nutzbar, erfordert der Zugriff auf erweiterte Funktionen wie mobile Synchronisation, 
  DVR oder Premium-Apps ein kostenpflichtiges Abonnement (\emph{Emby Premiere}). 
  Ursprünglich als Open-Source-Projekt gestartet, wechselte Emby später in eine proprietäre Lizenz. 
  Damit ist die Lösung für Privatanwender attraktiv, während sie im institutionellen Umfeld aufgrund der Kosten- 
  und Lizenzstruktur weniger flexibel erscheint.\footcite[Vgl.][]{wikipedia_emby}\footcite[Vgl.][]{emby_docs}
  \\
  \newline
  \textbf{Jellyfin:}
  Jellyfin ist eine vollständig freie und quelloffene Medienserver-Software, die als Community-getragene Alternative zu Plex und Emby entwickelt wird. 
  Sie eignet sich besonders für Szenarien, in denen Transparenz, Datenschutz 
  und Unabhängigkeit von kommerziellen Modellen eine Rolle spielen, wie etwa in Hochschulen oder öffentlichen Bibliotheken. 
  Jellyfin unterstützt die Verwaltung und das Streaming von Audio-, Video- und Bilddateien und bietet Clients für Webbrowser, Mobilgeräte und Smart-TVs. 
  Alle Funktionen, einschließlich Transkodierung und Multiuser-Verwaltung, 
  sind ohne Einschränkung verfügbar.\footcite[Vgl.][]{wikipedia_jellyfin}\footcite[Vgl.][]{jellyfin_docs}
  \\
  \newline  
  \textbf{VLC-Streaming:}
  Der \emph{VLC Media Player} ist eine weit verbreitete, freie und quelloffene Software der \emph{VideoLAN}-Organisation. 
  Neben der Wiedergabe nahezu aller gängigen Audio- und Videoformate umfasst VLC auch leistungsfähige Streaming-Funktionen. 
  Inhalte können beispielsweise über HTTP, RTP oder RTSP verteilt werden, was den VLC für den Aufbau kleiner Streaming-Server, 
  Testumgebungen oder den schnellen Medientransfer prädestiniert. 
  Dank seiner Flexibilität und Plattformunabhängigkeit findet VLC Anwendung in Bildungseinrichtungen ebenso wie im privaten Bereich.\footcite[Vgl.][]{wikipedia_vlc}\footcite[Vgl.][]{vlc_docs}  
  \\
  \newline
  \textbf{Icecast:}
  \emph{Icecast}, entwickelt von der \emph{Xiph.org Foundation}, ist eine freie Open-Source-Software zur Einrichtung von Streaming-Servern. 
  Hauptsächlich im Audiobereich (z. B. Internetradios) eingesetzt, unterstützt Icecast auch Videostreams. 
  Neben offenen Formaten wie Ogg Vorbis, Opus und Theora können auch verbreitete Standards wie MP3 oder AAC genutzt werden. 
  Durch seine Stabilität und Skalierbarkeit eignet sich Icecast besonders für Webradios, nichtkommerzielle Sender sowie für Bildungs- und Forschungseinrichtungen, 
  in denen offene Standards von zentraler Bedeutung sind.\footcite[Vgl.][]{wikipedia_icecast}\footcite[Vgl.][]{icecast_docs}  
  \\
  \newline
  \textbf{DLNA-basierte Systeme:}
  \emph{DLNA} (Digital Living Network Alliance) ist ein Industriestandard, der die Interoperabilität zwischen Geräten im Heimnetzwerk sicherstellt. 
  DLNA-konforme Server stellen Inhalte wie Musik, Videos und Bilder bereit, die von kompatiblen Clients 
  (Smart-TVs, Spielkonsolen, Streaming-Boxen oder mobilen Geräten) abgerufen werden können. 
  Typische Implementierungen sind \emph{MiniDLNA/ReadyMedia} oder in NAS-Systeme integrierte Medienserver. 
  Während DLNA im privaten Bereich nach wie vor verbreitet ist, verliert es im professionellen Umfeld zunehmend an Relevanz, da moderne, HTTP-basierte Verfahren wie HLS oder DASH flexiblere Möglichkeiten bieten.\footcite[Vgl.][]{wikipedia_dlna}\footcite[Vgl.][]{dlna_specs} 

\subsection{Datenformate und Standards}  
Für den Betrieb eines Medienservers ist die Unterstützung gängiger Datenformate entscheidend, 
da diese maßgeblich die Kompatibilität mit Endgeräten sowie die Qualität und Effizienz der Speicherung und Übertragung bestimmen.  

\textbf{Audioformate:}  
Im Audiobereich haben sich unterschiedliche Standards etabliert, die sich hinsichtlich Qualität, Kompression und Einsatzgebiet unterscheiden.  
Das MP3-Format (\emph{MPEG-1/2 Audio Layer III}) ist verlustbehaftet komprimiert und gilt als de facto Standard im Musikbereich, 
insbesondere wegen seiner geringen Dateigrößen und breiten Unterstützung durch Endgeräte.\footcite[Vgl.][S.~702~f.]{tanenbaum_computernetworks}  
Für verlustfreie Archivierung wird häufig FLAC (\emph{Free Lossless Audio Codec}) genutzt, 
das bei identischer Klangqualität wie das Original eine deutliche Speicherreduktion ermöglicht.  
AAC (\emph{Advanced Audio Coding}), ebenfalls verlustbehaftet und von der MPEG-Gruppe spezifiziert, 
wird heute von vielen Streamingdiensten und Plattformen (z.\,B. YouTube, iTunes, Spotify) als Standard eingesetzt.\footcite[Vgl.][]{openlearn_aac} 
WAV (\emph{Waveform Audio File Format}) speichert Audiodaten unkomprimiert, was höchste Qualität gewährleistet, jedoch mit hohem Speicherbedarf einhergeht.
Dieses Format findet daher primär in professionellen Produktionsumgebungen oder für kurzfristige Verarbeitungsschritte Anwendung.\footcite[Vgl.][]{wikipedia_wav}  

\textbf{Containerformate:}  
Neben reinen Audioformaten sind Containerformate für die Organisation komplexerer Inhalte entscheidend.  
Das Matroska-Format (MKV) ist ein freier, erweiterbarer Standard, der neben Audio- und Videoströmen auch Untertitel, 
Kapitelinformationen und mehrere Tonspuren in einer Datei vereinen kann.\footcite[Vgl.][siehe 1. Introduction]{rfc9559}  
MP4, ein ISO-Standard auf Basis von \emph{MPEG-4 Part 14}, ist weltweit eines der am weitesten verbreiteten Containerformate 
und bildet insbesondere die Grundlage für Video-Streaming über HTTP. 
Es überzeugt durch hohe Kompatibilität mit Abspielgeräten und Streaming-Plattformen.\footcite[Vgl.][S.~702~f.]{tanenbaum_computernetworks}

\textbf{Metadatenstandards:}
Metadaten („Daten über Daten“) beschreiben Eigenschaften von Mediendateien und ermöglichen deren effiziente Suche, Auffindbarkeit und Verwaltung. 
Ein weit verbreiteter Standard in MP3-Dateien sind ID3-Tags. 
Diese Container erlauben das Einbetten von Informationen wie Titel, Interpret, Album, Genre oder Coverbilder direkt in die Datei.\footcite[Vgl.][siehe 4. Declared ID3v2 frames]{id3org_spec}

Ein generischer, interdisziplinärer Standard ist Dublin Core. Dieser umfasst eine Menge von Kernfeldern (z. B. Title, Creator, Subject, Date, Format, Rights), 
die zur Beschreibung digitaler Ressourcen genutzt werden können\footcite[Vgl.][siehe The Elements]{dublincore_set}. 
Seine Stärke liegt in der Interoperabilität: Dublin Core wird von vielen Bibliotheks-, Museums- und Archivsystemen unterstützt.\footcite[Vgl.][siehe 1.1. What is Metadata?]{dublincore_using}

Daneben gibt es weitere Metadatenstandards, die speziellere Anforderungen adressieren. 
Beispiele sind METS (Metadata Encoding and Transmission Standard), MODS (Metadata Object Description Schema) 
oder MARCXML, die in Bibliotheks- und Archivumgebungen zusätzlich verwendet werden, 
insbesondere wenn detailliertere bibliografische Informationen benötigt werden.\footcite[Vgl.][]{loc_mets} \footcite[Vgl.][]{loc_mods}  
 
\subsection{Rechtsfragen}  
Neben technischen Aspekten spielen rechtliche Rahmenbedingungen eine zentrale Rolle beim Betrieb von Medienservern.  

\textbf{Lizenzen und Lizenzmodelle:}  
Die Wahl der eingesetzten Software und Codecs ist auch aus lizenzrechtlicher Sicht relevant.  
Freie Software wie Jellyfin unterliegt in der Regel der \emph{GNU General Public License (GPL)}, die die Nutzung, Modifikation und Weitergabe des Quellcodes gestattet, 
solange abgeleitete Werke unter denselben Bedingungen veröffentlicht werden.\footcite[Vgl.][GNU General Public License, Version~3, 2007]{gnu_gplv3}  
Kommerzielle Systeme wie Plex oder Emby hingegen stehen unter proprietären Lizenzen, die Nutzung, Verbreitung und Anpassung einschränken.  
Darüber hinaus können auch bei der Verwendung von Codecs Lizenzgebühren anfallen: So ist AAC (\emph{Advanced Audio Coding}) durch MPEG-LA patentiert 
und unterliegt Lizenzkosten bei bestimmter kommerzieller Nutzung, ebenso H.264/AVC.\footcite[Vgl.][MPEG Licensing Authority, 2025]{mpeg_la}  

\textbf{Urheberrechtliche Rahmenbedingungen:}  
Die Bereitstellung von Medieninhalten wird maßgeblich durch das Urheberrechtsgesetz (UrhG) geregelt.  
Nach §~53 UrhG sind Privatkopien zwar grundsätzlich erlaubt, jedoch nur im engen persönlichen Umfeld.  
Eine öffentliche Zugänglichmachung nach §~19a UrhG - wie sie bei einem Medienserver innerhalb einer Organisation oder Bibliothek vorliegt -
setzt hingegen entsprechende Nutzungsrechte oder Lizenzen der Rechteinhaber voraus.\footcite[Vgl.][§§~53, 19a UrhG]{urhg2025}  

\textbf{Bibliotheksrecht im Hochschulkontext:}  
Für Hochschulbibliotheken gelten Sonderregelungen.  
§~60a UrhG erlaubt die Nutzung urheberrechtlich geschützter Werke für Unterricht und Lehre, etwa in Kursen oder in einer Lernplattform.  
§~60e UrhG regelt darüber hinaus Nutzungen durch Bibliotheken, beispielsweise die Bereitstellung von Werken in elektronischer Form für Forschungs- und Unterrichtszwecke.  
Dennoch müssen in der Praxis bestehende Lizenzverträge und Verlagsbedingungen beachtet werden, 
da gesetzliche Schrankenregelungen nicht alle Nutzungsformen abdecken.\footcite[Vgl.][§§~60a, 60e UrhG]{urhg2025}  

\section{Medienserver im Bibliothekskontext}  
Der Einsatz von Medienservern in Bibliotheken eröffnet neue Möglichkeiten für den Zugang zu digitalen Sammlungen, bringt aber auch besondere Anforderungen mit sich - technisch, rechtlich und hinsichtlich Langzeitverfügbarkeit.

  \subsection{Digitalisierung von Beständen}  
  Bibliotheken besitzen oft analoge oder hybride Bestände - etwa Audio- und Videomedien, Filmarchive oder Tonträger -, die durch Digitalisierung langfristig gesichert und zugänglich gemacht werden sollen. Der Prozess gliedert sich idealerweise in vier Phasen:

  \begin{itemize}
    \item \textbf{Auswahl und Priorisierung:}  
      Nicht alle Bestände können sofort digitalisiert werden. Kriterien für eine sinnvolle Auswahl sind u. a. der Erhaltungszustand, die Nutzungshäufigkeit, die rechtliche Situation sowie historische oder kulturelle Bedeutung. Institutionelle Richtlinien oder internationale Leitlinien wie UNESCO/PERSIST geben dazu Orientierung.:contentReference[oaicite:3]{index=3}  
      Auch Bibliotheken wie die University of Cincinnati verwenden solche Kriterien in ihren Digitization-Selection-Guidelines.:contentReference[oaicite:4]{index=4}  

    \item \textbf{Erfassung und Digitalisierung:}  
      Die Objekte werden mit geeigneten Geräten gescannt oder digital abgewickelt (z. B. Scanner, Filmabtaster oder spezielle Audiogeräte). Wichtig ist, dass die digitalen Dateien eine hohe Qualität (z. B. Auflösung, Farbtiefe, Samplingrate) erreichen, sodass sie als Masterversionen für spätere Ableitungen dienen können.  
      Die Smithsonian Institution nennt als Standardformat z. B. das Broadcast-Wave-Format (BWF) bei Audio mit 24 Bit / 96 kHz für Archivqualität.:contentReference[oaicite:5]{index=5}  
      Für Bild-/Foto-Digitalisate gelten die FADGI-Technischen Richtlinien als international anerkannter Referenzrahmen, insbesondere hinsichtlich Bildqualität, Metadaten und Workflow-Controlling.:contentReference[oaicite:6]{index=6}  

    \item \textbf{Nachbearbeitung und Metadatenanreicherung:}  
      Nach der Erfassung werden die Rohdaten optimiert (z. B. Rauschfilter, Farbkorrektur) und in normative Formate überführt. Parallel dazu werden Metadaten eingepflegt — technisch, beschreibend und strukturell —, um Auffindbarkeit, Verwaltung und Langzeitnutzung zu ermöglichen.

    \item \textbf{Ableitung von Zugriffsvarianten:}  
      Für den täglichen Zugriff werden üblicherweise abgeleitete Versionen (z. B. mit kleinerer Auflösung oder effizienteren Codecs) erzeugt, während die Masterdateien sicher archiviert bleiben.
  \end{itemize}

  Diese digitalen Prozesse orientieren sich an international anerkannten Standards, wie den FADGI-Richtlinien oder Metamorfoze, und an Best Practices in der digitalen Archivierung und Bibliothekenarbeit.:contentReference[oaicite:7]{index=7}  

\subsection{Zugriff und Rechteverwaltung}  
Im Bibliothekskontext bestehen oft restriktive Anforderungen an den Zugriff auf Medien:

\begin{itemize}
  \item \textbf{Eingeschränkter Zugang:} Der Zugang zu digitalisierten Medien erfolgt häufig nur innerhalb des Hochschulnetzes oder über VPN-Verbindungen, um urheberrechtliche Beschränkungen und Lizenzvereinbarungen zu wahren.  
  \item \textbf{Authentifizierung und Autorisierung:} Bibliotheken nutzen bestehende Identity-Management-Systeme (z. B. Shibboleth, LDAP, OpenID Connect), um sicherzustellen, dass nur berechtigte Nutzer (Studenten, Mitarbeitende) Zugriff haben.  
  \item \textbf{Digitale Rechteverwaltung (DRM) / Lizenzsteuerung:} In manchen Fällen sind Inhalte lizenzpflichtig und müssen mit DRM-Technologien geschützt werden oder der Zugriff auf bestimmte Zeitfenster begrenzt sein.  
  \item \textbf{Auditierung und Protokollierung:} Jede Nutzung (z. B. Streamabruf, Download) sollte protokolliert werden, um Vertragsbedingungen zu überwachen, Missbrauch zu erkennen und Nutzungsstatistiken zu führen.  
\end{itemize}

Diese Aspekte machen Bibliotheks-Medienserver anspruchsvoller als einfache Streaming-Installationen für private Nutzer.

\subsection{Nachhaltigkeit und Langzeitarchivierung}  
Die Sicherung digitaler Medien über Jahrzehnte hinweg erfordert robuste Strategien:

\begin{itemize}
  \item \textbf{Redundanz und Verteilung (3-2-1 Regel):} Mindestens drei Kopien, auf zwei unterschiedlichen Speichermedien und eine Kopie an einem externen Standort. \footcite[Vgl.][„3-2-1 Rule“]{turn0search7}  
  \item \textbf{Migration und Formatpflege:} Dateiformate und Speichermedien verändern sich. Archivbestände müssen regelmäßig überprüft und falls nötig in aktuelle, verlustfreie Formate migriert werden.  
  \item \textbf{Integritätsprüfung und Fixity-Checks:} Regelmäßige Checksummenprüfungen und Monitoring, um bitweise Konsistenz und Unversehrtheit der Daten sicherzustellen.  
  \item \textbf{Cloud- vs. On-Premises-Lösungen:} Bibliotheken wägen zwischen eigener Infrastruktur und Cloud-Diensten ab. Cloud erlaubt Skalierbarkeit und Redundanz, bringt aber Aspekte wie Datenhoheit, Kosten und Anbieterabhängigkeit mit sich. \footcite[Vgl.][„Cloud Services in Digital Preservation“]{turn0search1}\footcite[Vgl.][S. ?]{turn0search9}  
  \item \textbf{Netzwerk der Kooperationen:} Verbünde, Konsortien und Netzwerke (z. B. MetaArchive Cooperative) ermöglichen verteilte Replikation, Format-Migration im Verbund und Schutz gegenüber Singularitätsrisiken. \footcite[Vgl.][]{turn0search21}  
\end{itemize}

\subsection{Lokale vs.\ Cloud-basierte Lösungen}  
Beim Vergleich zwischen lokalen und cloudbasierten Medienservern im Bibliothekskontext ergeben sich folgende Vor- und Nachteile:

\textbf{Lokale Lösungen}  

\begin{itemize}
  \item volle Kontrolle über Hardware, Daten und Sicherheit  
  \item keine laufenden Speicher- oder Transferkosten an Drittanbieter  
  \item abhängig von eigener Infrastruktur und personellen Ressourcen  
  \item Risiko von Hardwarealterung, Ausfällen oder Standortschäden  
\end{itemize}

\textbf{Cloud-basierte Ansätze}  

\begin{itemize}
  \item hohe Skalierbarkeit, Redundanz und geografische Verteilung  
  \item Backups und Disaster Recovery als Dienstleistung  
  \item laufende Kosten, mögliche Datenübertragungs- und Auslagerungskosten  
  \item Datenschutz- und Rechtsaspekte (Datenhoheit, Standort der Server) \footcite[Vgl.][„Digital Preservation and the Cloud“]{turn0search3}  
  \item Abhängigkeit vom Anbieter, potenzieller Lock-in  
\end{itemize}

\textbf{Besonderer bibliothekarischer Kontext:}  
Für Bibliotheken kann Hybridbetrieb (Teile lokal, Teile in der Cloud) sinnvoll sein: z. B. Masterarchive lokal, weniger kritische Katalogversionen in der Cloud. Ebenso kann ein gestaffelter Zugriff sinnvoll sein: interne Zugriffe lokal, externe Benutzer über Cloud-Endpunkte.


% ------ Praktische Umsetzung ------
\chapter*{Praktische Umsetzung}
\addcontentsline{toc}{chapter}{Praktische Umsetzung}
\setcounter{section}{0}

\section{Zielsetzung}
  \subsection{Funktionalität}
  \subsection{Benutzerfreundlichkeit}
  \subsection{Wartbarkeit}
  \subsection{Sicherheit}
  \subsection{Rechtliche Rahmenbedingungen}
\section{Vorraussetzungen}
  \subsection{Hardware}
  \subsection{Software}
  \subsection{Netzwerk}
  \subsection{Rechtliche Rahmenbedingungen}
\section{Installation und Konfiguration}  
Die Installation und Konfiguration der zentralen Softwarekomponenten stellt die Grundlage für den zuverlässigen und sicheren Betrieb des Medienservers dar.
Nun werden die benötigten Werkzeuge und Dienste vorgestellt und ihre Einrichtung anhand einer praxisorientierten Schritt-für-Schritt-Anleitung beschrieben.
Damit soll der Aufbau des Medienservers in der Bibliothek nachvollziehbar und reproduzierbar gemacht werden.

  \subsection{Ubuntu Server}  
  Ubuntu Server ist eine weit verbreitete Linux-Distribution, die durch ihre Stabilität, breite Community-Unterstützung und regelmäßige Sicherheitsupdates überzeugt.  
  Das aktuelle Installations-ISO kann von der offiziellen Webseite heruntergeladen werden: \url{https://ubuntu.com/download/server}.  
  Nach dem Download wird das Image auf einen USB-Stick geschrieben (z.\,B. mit \texttt{Rufus} oder \texttt{dd} unter Linux).  
  Dabei ist zu beachten, dass der Stick keine wichtigen Daten enthält, da dieser vollständig überschrieben wird, und eine Mindestgröße von 5\,GB haben sollte.    
  Um die Installation zu starten, wird der Stick angeschlossen und über das Boot-Menü (Taste \texttt{F12}, \texttt{Esc} oder \texttt{Del}, abhängig vom Hersteller) ausgewählt.  
  Die textbasierte Installationsroutine führt anschließend durch Partitionierung, Netzwerkkonfiguration und die Einrichtung eines Administratorkontos. 
  Dabei können SSH und Docker bereits mitinstalliert werden. Der USB-Stick kann danach entfernt und der Server neu gestartet werden.
  Nähere Informationen zu diesem Vorgang sind in der offiziellen Dokumentation verfügbar: \url{https://ubuntu.com/server/docs/installation}.

  \subsection{SSH}  
  \emph{Secure Shell (SSH)} ist das zentrale Werkzeug für den sicheren Fernzugriff auf den Server.  
  Der SSH-Server kann unter Ubuntu bereits bei der Installation aktiviert werden. 
  Ist dies nicht der Fall, läuft die nachträgliche Installation und Aktivierung wie folgt:  

  \begin{minted}[breaklines, linenos]{bash}
sudo apt update
sudo apt install openssh-server
sudo systemctl enable --now ssh
  \end{minted}  

  Der Zugriff erfolgt dann von einem Client aus über:  

  \begin{minted}[breaklines, linenos]{bash}
ssh benutzername@server-ip
  \end{minted}  

  Die offizielle Dokumentation befindet sich unter \url{https://www.openssh.com/}.  
  Für erhöhte Sicherheit wird empfohlen, die Passwort-Authentifizierung zu deaktivieren und stattdessen SSH-Schlüssel zu verwenden.  

  \subsubsection*{Wichtige Befehle}  

  \textbf{Service steuern:}  
  \begin{minted}[breaklines, linenos]{bash}
sudo systemctl start ssh
sudo systemctl stop ssh
sudo systemctl restart ssh
systemctl status ssh
  \end{minted}  

  \textbf{IP-Adresse des Servers ermitteln:}  
  \begin{minted}[breaklines, linenos]{bash}
ip a
  \end{minted}  

  \textbf{SSH-Keys erstellen und übertragen (empfohlen):}  
  \begin{minted}[breaklines, linenos]{bash}
ssh-keygen -t ed25519
ssh-copy-id benutzername@server-ip
  \end{minted}  

  \textbf{Dateien übertragen (scp):}  
  \begin{minted}[breaklines, linenos]{bash}
# Datei von Client auf Server kopieren
scp /pfad/zur/datei benutzername@server-ip:/ziel/pfad/

# Datei vom Server auf Client kopieren
scp benutzername@server-ip:/pfad/auf/server/datei /lokaler/pfad/
  \end{minted}  

  \textbf{Verzeichnisse synchronisieren (rsync):}  
  \begin{minted}[breaklines, linenos]{bash}
rsync -av benutzername@server-ip:/remote/verzeichnis/ ./lokal/
  \end{minted}  

  \subsection{UFW}  
  Die \emph{Uncomplicated Firewall (UFW)} dient zur einfachen Verwaltung der Firewall-Regeln.  
  UFW ist in den Ubuntu-Repositories enthalten und kann wie folgt installiert und aktiviert werden:  
  \begin{verbatim}
  sudo apt install ufw
  sudo ufw enable
  \end{verbatim}  
  Anschließend lassen sich Regeln hinzufügen, z.\,B. um SSH und HTTPS freizuschalten:  
  \begin{verbatim}
  sudo ufw allow 22/tcp
  sudo ufw allow 443/tcp
  \end{verbatim}  
  Alle weiteren Ports bleiben standardmäßig blockiert.  
  Offizielle Informationen sind in der Ubuntu-Dokumentation verfügbar: \url{https://help.ubuntu.com/community/UFW}.\footcite[Vgl.][Kapitel~22, S.~932~ff.]{nemeth_unixlinux}

  \subsection{Docker}  
  Docker ist eine Container-Plattform, die Anwendungen in isolierten Umgebungen ausführt. 
  Falls Docker nicht bei der Ubuntu-Installation mitinstalliert wurde, kann die Installation komfortabel über ein vorbereitetes Skript erfolgen. 
  Die offizielle Installationsanleitung ist verfügbar unter: \url{https://docs.docker.com/engine/install/}.  
  Dort findet sich auch der Inhalt des Skriptes \texttt{install-docker.sh}. Dieser sollte vor der Installation auf die neueste Version überprüft werden. 
  Ein Beispielskript (Stand 24. September 2025) findet sich im GitHub-Repository: \url{https://github.com/RathgebL/Bachelor_Thesis-New_Mediaserver}.
  Dieses enthält alle notwendigen Befehle zum Einrichten des Docker-Repositories und zur Installation der Docker-Pakete.
  Der Installationsvorgang gliedert sich nun in folgende Schritte:  

  \begin{enumerate}
    \item Datenträger mit Skript verfügbar machen:
    \begin{minted}[breaklines, linenos]{bash}
# Stick anschließen und Gerät ermitteln (z.B. sdb1)
lsblk

# Output könnte z.B. so aussehen:
NAME   MAJ:MIN RM   SIZE RO TYPE MOUNTPOINTS
sda      8:0    0 238.5G  0 disk
  sda1   8:1    0   512M  0 part /boot/efi
  sda2   8:2    0   238G  0 part /
sdb      8:16   1  14.9G  0 disk
  sdb1   8:17   1  14.9G  0 part

# Mountpoint erstellen (falls noch nicht vorhanden)
sudo mkdir -p /mnt/usb

# USB-Stick einhängen (Beispiel: sdb1)
sudo mount /dev/sdb1 /mnt/usb

# In das Verzeichnis des USB-Sticks wechseln
cd /mnt/usb
    \end{minted}

    \item Skript ausführbar machen und starten:
    \begin{minted}[breaklines, linenos]{bash}
chmod +x install-docker.sh
./install-docker.sh
    \end{minted}

    \item Überprüfen der Installation:
    \begin{minted}[breaklines, linenos]{bash}
# Verifiziere, dass der Docker-Dienst läuft
sudo systemctl status docker

# Testen der Installation mit dem Hello-World-Container
sudo docker run hello-world
    \end{minted}
  
    \item Nutzung:
    Nach der Installation kann Docker Compose für die Verwaltung von Containern genutzt werden.  
    Dafür ist es sehr nützlich folgende Befehle, die im Verzeichnis ausgeführt werden, 
    in dem sich die \texttt{docker-compose.yml}-Datei befindet, zu kennen:

    \begin{minted}[breaklines, linenos]{bash}
# Container erstellen und im Hintergrund starten
sudo docker compose up -d

# Container starten und stoppen
sudo docker compose start
sudo docker compose stop

# Änderungen übernehmen (z.B. nach Anpassung der Compose-Datei)
sudo docker compose up -d --build

# Container-Status überprüfen
sudo docker compose ps
sudo docker ps

# Container beenden und entfernen
sudo docker compose down
    \end{minted}
  \end{enumerate}

  \subsection{nginx}  
  \emph{nginx} ist ein moderner Webserver und Reverse Proxy, der sich durch hohe Performance auszeichnet.  
  Die offizielle Website bietet Installationshinweise und Downloads: \url{https://nginx.org/en/download.html}.  
  Unter Ubuntu kann nginx direkt über die Paketverwaltung installiert werden:

  \begin{minted}[breaklines, linenos]{bash}
# Paketliste aktualisieren
sudo apt update

# nginx installieren
sudo apt install nginx -y

# nginx starten und beim Systemstart aktivieren
sudo systemctl enable --now nginx

# Status überprüfen
sudo systemctl status nginx

# Test im Browser oder mit curl (sollte "Welcome to nginx!" zurückgeben)
curl http://localhost
  \end{minted} 

  Die Konfiguration erfolgt über Dateien im Verzeichnis \texttt{/etc/nginx/sites-available/}, 
  die durch Symlinks nach \texttt{/etc/nginx/sites-enabled/} aktiviert werden.  
  Für den Medienserver empfiehlt sich eine Konfiguration wie im folgenden Beispiel (als txt-Datei auch im GitHub-Repository verfügbar):  

  \begin{minted}[breaklines, linenos]{nginx}
server {
    listen 80;
    server_name mediaserver.local;

    # Booklets bereitstellen
    location /booklets/ {
        alias /home/user/srv/booklets/;
        autoindex off;              # kein Verzeichnislisting
        autoindex_exact_size off;
        autoindex_localtime on;
    }

    # Weiterleitung an Navidrome
    location / {
        proxy_pass         http://127.0.0.1:4533;
        proxy_http_version 1.1;
        proxy_set_header   Upgrade $http_upgrade;
        proxy_set_header   Connection "upgrade";
        proxy_set_header   Host $host;
        proxy_set_header   X-Real-IP $remote_addr;
        proxy_set_header   X-Forwarded-For $proxy_add_x_forwarded_for;
        proxy_set_header   X-Forwarded-Proto $scheme;
        proxy_redirect     off;
    }
}
  \end{minted}  

  Die Datei wird in \texttt{/etc/nginx/sites-available/mediaserver} abgelegt 
  und über einen Symlink aktiviert:  

  \begin{minted}[breaklines, linenos]{bash}
sudo ln -s /etc/nginx/sites-available/mediaserver \
           /etc/nginx/sites-enabled/
sudo nginx -t
sudo systemctl reload nginx
  \end{minted}  

  Es folgen einige nützliche Befehle zur Verwaltung von nginx:
  \textbf{Service steuern}  
  \begin{minted}[breaklines, linenos]{bash}
sudo nginx -t                  # Konfiguration testen
sudo systemctl start nginx     # starten
sudo systemctl stop nginx      # stoppen
sudo systemctl restart nginx   # neu starten
sudo systemctl reload nginx    # nur Konfiguration neu laden
  \end{minted}

  \textbf{Status prüfen}  
  \begin{minted}[breaklines, linenos]{bash}
systemctl status nginx
  \end{minted}

  \textbf{Logs ansehen}  
  \begin{minted}[breaklines, linenos]{bash}
# Zugriffe (wer greift auf was zu)
tail -f /var/log/nginx/access.log

# Fehler (z. B. falsche Pfade oder Syntaxfehler in Config)
tail -f /var/log/nginx/error.log
  \end{minted}
 
  \subsection{Navidrome}  
  Navidrome ist ein leichtgewichtiger Musikserver, der als Docker-Container besonders einfach bereitgestellt werden kann.  
  Die offizielle Projektseite mit Anleitungen und Docker-Images befindet sich unter: \url{https://www.navidrome.org/docs/}.  
  Ein typisches \texttt{docker-compose.yml}-Beispiel für Navidrome sieht folgendermaßen aus:  
  \begin{verbatim}
  version: "3"
  services:
    navidrome:
      image: deluan/navidrome:latest
      restart: unless-stopped
      ports:
        - "4533:4533"
      volumes:
        - "./data:/data"
        - "./music:/music:ro"
  \end{verbatim}   

\section{Integration in die bestehende IT-Infrastruktur}
  \subsection{Netzwerk-Anbindung}  
  \subsection{Datenbank anbinden}
    \subsubsection{Datenstruktur}
    Benennungskonvention
  \subsubsection{Dateiformate}
  \subsubsection{Metadaten}
    \begin{itemize}
      \item ID3-Tags
      \item Vorbis Comments
      \item FLAC-Metadaten
      \item MusicBrainz
    \end{itemize}
  \subsubsection{Datenverwaltung}
    Datenpflege
    Backup-Strategie
  \subsubsection{Datenquelle}
    Skript zur Digitalisierung
    Skript zur Metadatenanreicherung
    Externe Quellen
  \subsubsection{Datenimport}
\section{Benutzerverwaltung und Zugriffsrechte}
  \subsection{Nutzergruppen definieren}
\section{Testen und Fehlerbehebung}
  \subsection{Funktionstests}
  \subsection{Problemlösung}   


\chapter*{Fazit}
\addcontentsline{toc}{chapter}{Fazit}
\setcounter{section}{0}
% Fazit folgt ...

\section{Ausblick}
% Ausblick folgt ...}

% ------ Notizen (am Ende entfernen!) ------
\section{Notizen}

\begin{enumerate}
  \item Server-Setup
    \begin{itemize}
      \item Rechner mit Ubuntu (24.04.3 LTS, da längerer Support) als Betriebssystem aufsetzen.
        \begin{itemize}
          \item Download von \url{https://ubuntu.com/download/server} auf USB-Stick
          \item Stick einstecken, Rechner booten und Installation starten
            \begin{itemize}
              \item dafür beim Anschalten des Rechners \texttt{Shift} drücken
              \item darauf achten, dass der Rechner in Zukunft nicht über USB bootet
              \item Rechner per LAN-Kabel mit dem Netzwerk verbinden
            \end{itemize}
        \end{itemize}

      \item Docker installieren
        \begin{itemize}
          \item Quickinstall:
          \begin{minted}[linenos]{bash}
sudo apt update
sudo apt install docker.io
sudo systemctl start docker
sudo systemctl enable docker
          \end{minted}

          \item Von Docker-Webseite (\url{https://docs.docker.com/engine/install/ubuntu/}):
          \begin{minted}[breaklines, linenos]{bash}
# Add Docker's official GPG key:
sudo apt-get update
sudo apt-get install ca-certificates curl
sudo install -m 0755 -d /etc/apt/keyrings
sudo curl -fsSL https://download.docker.com/linux/ubuntu/gpg -o /etc/apt/keyrings/docker.asc
sudo chmod a+r /etc/apt/keyrings/docker.asc

# Add the repository to Apt sources:
echo \
"deb [arch=$(dpkg --print-architecture) signed-by=/etc/apt/keyrings/docker.asc] https://download.docker.com/linux/ubuntu \
$(. /etc/os-release && echo "${UBUNTU_CODENAME:-$VERSION_CODENAME}") stable" | \
sudo tee /etc/apt/sources.list.d/docker.list > /dev/null
sudo apt-get update
          \end{minted}

          \item Test mit:
          \begin{minted}[linenos]{bash}
sudo docker run hello-world
          \end{minted}
        \end{itemize}
    \end{itemize}

  \item Auswahl und Installation der Medienserver-Software
    \begin{itemize}
      \item Hier kommen die Softwareoptionen rein
      \item Installationsschritte …
    \end{itemize}

  \item Integration mit der bestehenden IT-Infrastruktur
    \begin{itemize}
      \item Netzwerk-Anbindung
      \item Datenbanken anbinden …
    \end{itemize}

  \item Benutzerverwaltung und Zugriffsrechte
    \begin{itemize}
      \item Nutzergruppen definieren
      \item Rechteverwaltung testen …
    \end{itemize}

  \item Testen und Fehlerbehebung
    \begin{itemize}
      \item Funktionstests
      \item Lasttests
      \item Bugfixing dokumentieren …
    \end{itemize}
\end{enumerate}
\newpage

\chapter*{Literaturverzeichnis}
\addcontentsline{toc}{chapter}{Literaturverzeichnis}
\printbibliography[heading=none]

\end{document}
