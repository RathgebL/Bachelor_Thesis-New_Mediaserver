\documentclass[12pt,a4paper]{article}

\usepackage[utf8]{inputenc}
\usepackage[T1]{fontenc}
\usepackage[german]{babel}
\usepackage{minted}
\usepackage{csquotes}
\usepackage{graphicx}
\usepackage{hyperref}
\usepackage[backend=biber,style=authoryear]{biblatex}

\addbibresource{bibliography.bib}

\begin{document}

% ------ Titelblatt ------
\begin{titlepage}
    \centering
    {\Large Hochschule für Musik Karlsruhe \\ [1em]}
    {\large Institut für Musikinformatik und Musikwissenschaft \\[6em]}

    {\Large \textbf{Bachelorarbeit} \\[2em]}

    {\LARGE \textbf{Aufsetzen eines neuen Medienservers für die Bibliothek der Hochschule für Musik Karlsruhe} \\[6em]}

    \begin{minipage}{0.9\textwidth}
        \raggedright
        \textbf{Autor:} Lennart Rathgeb \\
        \textbf{Matrikelnummer:} 13883 \\
        \textbf{Adresse:} Insterburger Straße 2, 76139 Karlsruhe \\
        \textbf{Studiengang:} Musikinformatik (HF), Musikwissenschaft (EF) \\
        \textbf{Erstleser:} Prof. Dr. Christoph Seibert \\
        \textbf{Zweitleser:} Daniel Höpfner \\
    \end{minipage}

    \vfill
    Karlsruhe, den \today
\end{titlepage}

% ------ Inhaltsverzeichnis ------
\newpage
\tableofcontents
\newpage

% ------ Abschnitte ------
\section{Notizen}

\begin{enumerate}
  \item Server-Setup
    \begin{itemize}
      \item Rechner mit Ubuntu (24.04.3 LTS, da längerer Support) als Betriebssystem aufsetzen.
        \begin{itemize}
          \item Download von \url{https://ubuntu.com/download/server} auf USB-Stick
          \item Stick einstecken, Rechner booten und Installation starten
            \begin{itemize}
              \item dafür beim Anschalten des Rechners \texttt{Shift} drücken
              \item darauf achten, dass der Rechner in Zukunft nicht über USB bootet
              \item Rechner per LAN-Kabel mit dem Netzwerk verbinden
            \end{itemize}
        \end{itemize}

      \item Docker installieren
        \begin{itemize}
          \item Quickinstall:
          \begin{minted}[linenos]{bash}
sudo apt update
sudo apt install docker.io
sudo systemctl start docker
sudo systemctl enable docker
          \end{minted}

          \item Von Docker-Webseite (\url{https://docs.docker.com/engine/install/ubuntu/}):
          \begin{minted}[breaklines, linenos]{bash}
# Add Docker's official GPG key:
sudo apt-get update
sudo apt-get install ca-certificates curl
sudo install -m 0755 -d /etc/apt/keyrings
sudo curl -fsSL https://download.docker.com/linux/ubuntu/gpg -o /etc/apt/keyrings/docker.asc
sudo chmod a+r /etc/apt/keyrings/docker.asc

# Add the repository to Apt sources:
echo \
"deb [arch=$(dpkg --print-architecture) signed-by=/etc/apt/keyrings/docker.asc] https://download.docker.com/linux/ubuntu \
$(. /etc/os-release && echo "${UBUNTU_CODENAME:-$VERSION_CODENAME}") stable" | \
sudo tee /etc/apt/sources.list.d/docker.list > /dev/null
sudo apt-get update
          \end{minted}

          \item Test mit:
          \begin{minted}[linenos]{bash}
sudo docker run hello-world
          \end{minted}
        \end{itemize}
    \end{itemize}

  \item Auswahl und Installation der Medienserver-Software
    \begin{itemize}
      \item Hier kommen die Softwareoptionen rein
      \item Installationsschritte …
    \end{itemize}

  \item Integration mit der bestehenden IT-Infrastruktur
    \begin{itemize}
      \item Netzwerk-Anbindung
      \item Datenbanken anbinden …
    \end{itemize}

  \item Benutzerverwaltung und Zugriffsrechte
    \begin{itemize}
      \item Nutzergruppen definieren
      \item Rechteverwaltung testen …
    \end{itemize}

  \item Testen und Fehlerbehebung
    \begin{itemize}
      \item Funktionstests
      \item Lasttests
      \item Bugfixing dokumentieren …
    \end{itemize}
\end{enumerate}
\newpage

\section{Einleitung}
% Einleitung folgt ...

\section{Implementierung}
\subsection{Vorraussetzungen}
\subsection{Sicherheit}

\section{Technischer Hintergrund}
- Medienserver
- Docker
- Ubuntu Server
- Navidrome
- NAS

\section{Fazit}
% Fazit folgt ...


% Beispiel für Zitation
Wie in \textcite{kuster2005musikinformatik} beschrieben, ...
Weitere Informationen finden sich bei Navidrome \parencite{navidrome2024}.

\section{Literaturverzeichnis}
%Literaturverzeichnis folgt ...

\printbibliography

\end{document}
