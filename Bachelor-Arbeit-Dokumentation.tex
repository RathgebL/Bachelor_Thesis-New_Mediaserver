\documentclass[12pt,a4paper]{article}

\usepackage[utf8]{inputenc}
\usepackage[T1]{fontenc}
\usepackage[german]{babel}
\usepackage{minted}
\usepackage{csquotes}
\usepackage{graphicx}
\usepackage{hyperref}
\usepackage[backend=biber,style=authoryear]{biblatex}

\addbibresource{bibliography.bib}

\title{Aufsetzen eines neuen Medienservers für die Bibliothek der Hochschule für Musik Karlsruhe}
\author{Lennart Rathgeb}
\date{\today}

\begin{document}

\maketitle

\tableofcontents
\newpage

\section{Notizen}
1. Server-Setup
- Rechner mit Ubuntu (24.04.3 LTS, da längerer Support) als Betriebssystem aufsetzen.
    - download von https://ubuntu.com/download/server aud USB-Stick
    - Stick einstecken, Rechner booten und Installation starten
        - dafür beim Anschalten des Rechners shift drücken
        - darauf achten, dass der Rechner in Zukunft nicht über USB bootet
        - Rechner per LAN Kabel mit dem Netzwerk verbinden
- Docker installieren
    Quickinstall:
    \begin{minted}[linenos]{bash}
    sudo apt update
    sudo apt install docker.io
    sudo systemctl start docker
    sudo systemctl enable docker
    \end{minted}
    Von Docker Webseite (https://docs.docker.com/engine/install/ubuntu/):
    \begin{minted}[breaklines, linenos]{bash}
    # Add Docker's official GPG key:
    sudo apt-get update
    sudo apt-get install ca-certificates curl
    sudo install -m 0755 -d /etc/apt/keyrings
    sudo curl -fsSL https://download.docker.com/linux/ubuntu/gpg -o /etc/apt/keyrings/docker.asc
    sudo chmod a+r /etc/apt/keyrings/docker.asc

    # Add the repository to Apt sources:
    echo \
    "deb [arch=$(dpkg --print-architecture) signed-by=/etc/apt/keyrings/docker.asc] https://download.docker.com/linux/ubuntu \
    $(. /etc/os-release && echo "${UBUNTU_CODENAME:-$VERSION_CODENAME}") stable" | \
    sudo tee /etc/apt/sources.list.d/docker.list > /dev/null
    sudo apt-get update
    \end{minted}
    
    Test mit:
    \begin{minted}[linenos]{bash} 
    sudo docker run hello-world 
    \end{minted}
- 


2. Auswahl und Installation der Medienserver-Software

3. Integration mit der bestehenden IT-Infrastruktur

4. Benutzerverwaltung und Zugriffsrechte

5. Testen und Fehlerbehebung

\section{Einleitung}
% Einleitung folgt ...

\section{Implementierung}
1. Vorraussetzungen
2. Sicherheit
3. 

\section{Technischer Hintergrund}
- Medienserver
- Docker
- Ubuntu Server
- Navidrome
- NAS

\section{Fazit}
% Fazit folgt ...


% Beispiel für Zitation
Wie in \textcite{kuster2005musikinformatik} beschrieben, ...
Weitere Informationen finden sich bei Navidrome \parencite{navidrome2024}.

\section{Literaturverzeichnis}
%Literaturverzeichnis folgt ...

\printbibliography

\end{document}
